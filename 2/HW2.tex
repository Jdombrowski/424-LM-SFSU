%%%%%%%%%%%%%%%%%%%%%%%%%%%%%%%%%%%%%%%%%
% Structured General Purpose Assignment
% LaTeX Template
%
% This template has been downloaded from:
% http://www.latextemplates.com
%
% Original author:
% Ted Pavlic (http://www.tedpavlic.com)
%
% Note:
% The \lipsum[#] commands throughout this template generate dummy text
% to fill the template out. These commands should all be removed when 
% writing assignment content.
%
%%%%%%%%%%%%%%%%%%%%%%%%%%%%%%%%%%%%%%%%%

%----------------------------------------------------------------------------------------
%	PACKAGES AND OTHER DOCUMENT CONFIGURATIONS
%----------------------------------------------------------------------------------------

\documentclass{article}

\usepackage{fancyhdr} % Required for custom headers
\usepackage{lastpage} % Required to determine the last page for the footer
\usepackage{extramarks} % Required for headers and footers
\usepackage{graphicx} % Required to insert images

\usepackage{lipsum} % Used for inserting dummy 'Lorem ipsum' text into the template
\usepackage{amsmath}
% Margins
\topmargin=-0.45in
\evensidemargin=0in
\oddsidemargin=0in
\textwidth=6.5in
\textheight=9.0in
\headsep=0.25in 

\linespread{1.1} % Line spacing

% Set up the header and footer
\pagestyle{fancy}
\lhead{\hmwkAuthorName} % Top left header
\chead{\hmwkClass\ (\hmwkClassInstructor\ \hmwkClassTime): \hmwkTitle} % Top center header
\rhead{\firstxmark} % Top right header
\lfoot{\lastxmark} % Bottom left footer
\cfoot{} % Bottom center footer
\rfoot{Page\ \thepage\ of\ \pageref{LastPage}} % Bottom right footer
\renewcommand\headrulewidth{0.4pt} % Size of the header rule
\renewcommand\footrulewidth{0.4pt} % Size of the footer rule

\setlength\parindent{0pt} % Removes all indentation from paragraphs

%----------------------------------------------------------------------------------------
%	DOCUMENT STRUCTURE COMMANDS
%	Skip this unless you know what you're doing
%----------------------------------------------------------------------------------------

% Header and footer for when a page split occurs within a problem environment
\newcommand{\enterProblemHeader}[1]{
\nobreak\extramarks{#1}{#1 continued on next page\ldots}\nobreak
\nobreak\extramarks{#1 (continued)}{#1 continued on next page\ldots}\nobreak
}

% Header and footer for when a page split occurs between problem environments
\newcommand{\exitProblemHeader}[1]{
\nobreak\extramarks{#1 (continued)}{#1 continued on next page\ldots}\nobreak
\nobreak\extramarks{#1}{}\nobreak
}

\setcounter{secnumdepth}{0} % Removes default section numbers
\newcounter{homeworkProblemCounter} % Creates a counter to keep track of the number of problems


% PROBLEM 8
\newcommand{\homeworkProblemName}{}
\newenvironment{homeworkProblem}[1][Problem \arabic{homeworkProblemCounter}]{ % Makes a new environment called homeworkProblem which takes 1 argument (custom name) but the default is "Problem #"
\stepcounter{homeworkProblemCounter} % Increase counter for number of problems
\renewcommand{\homeworkProblemName}{#1} % Assign \homeworkProblemName the name of the problem
\section{\homeworkProblemName} % Make a section in the document with the custom problem count
\enterProblemHeader{\homeworkProblemName} % Header and footer within the environment
}{
\exitProblemHeader{\homeworkProblemName} % Header and footer after the environment
}

\newcommand{\problemAnswer}[1]{ % Defines the problem answer command with the content as the only argument
\noindent\framebox[\columnwidth][c]{\begin{minipage}{0.98\columnwidth}#1\end{minipage}} % Makes the box around the problem answer and puts the content inside
}

\newcommand{\homeworkSectionName}{}
\newenvironment{homeworkSection}[1]{ % New environment for sections within homework problems, takes 1 argument - the name of the section
\renewcommand{\homeworkSectionName}{#1} % Assign \homeworkSectionName to the name of the section from the environment argument
\subsection{\homeworkSectionName} % Make a subsection with the custom name of the subsection
\enterProblemHeader{\homeworkProblemName\ [\homeworkSectionName]} % Header and footer within the environment
}{
\enterProblemHeader{\homeworkProblemName} % Header and footer after the environment
}
   
%----------------------------------------------------------------------------------------
%	NAME AND CLASS SECTION
%----------------------------------------------------------------------------------------

\newcommand{\hmwkTitle}{Assignment\ \# 2 \ Chapter 3} % Assignment title
\newcommand{\hmwkDueDate}{Monday,\ October\ 2,\ 2017} % Due date
\newcommand{\hmwkClass}{MATH\ 424} % Course/class
\newcommand{\hmwkClassTime}{11:10am} % Class/lecture time
\newcommand{\hmwkClassInstructor}{Kafai} % Teacher/lecturer
\newcommand{\hmwkAuthorName}{Jonathan Dombrowski} % Your name

%----------------------------------------------------------------------------------------
%	TITLE PAGE
%----------------------------------------------------------------------------------------

\title{
\vspace{2in}
\textmd{\textbf{\hmwkClass:\ \hmwkTitle}}\\
\normalsize\vspace{0.1in}\small{Due\ on\ \hmwkDueDate}\\
\vspace{0.1in}\large{\textit{\hmwkClassInstructor\ \hmwkClassTime}}
\vspace{3in}
}

\author{\textbf{\hmwkAuthorName}}
\date{} % Insert date here if you want it to appear below your name

%----------------------------------------------------------------------------------------

\begin{document}

\maketitle

%----------------------------------------------------------------------------------------
%	TABLE OF CONTENTS
%----------------------------------------------------------------------------------------

%\setcounter{tocdepth}{1} % Uncomment this line if you don't want subsections listed in the ToC

\newpage
\tableofcontents
\newpage

%----------------------------------------------------------------------------------------
%	PROBLEM 8
%----------------------------------------------------------------------------------------

% To have just one problem per page, simply put a \clearpage after each problem



%----------------------------------------------------------------------------------------
%	PROBLEM 2
%----------------------------------------------------------------------------------------

\begin{homeworkProblem}[Question 8] % Custom section title
% Question

Predicting home sales price. Real estate investors,
homebuyers, and homeowners often use the
appraised (or market) value of a property as a
basis for predicting sale price. Data on sale prices
and total appraised values of 76 residential prop-
erties sold in 2008 in an upscale Tampa, Florida,
neighborhood named Tampa Palms are saved in
the TAMPALMS file. The first five and last five
observations of the data set are listed in the accom-
panying table.
\newline
\newline(a) Propose a straight-line model to relate the
appraised property value x to the sale price y
for residential properties in this neighborhood.
\newline(b) A MINITAB scatterplot of the data is shown
on the previous page. [Note: Both sale price
and total market value are shown in thousands
of dollars.] Does it appear that a straight-line
model will be an appropriate fit to the data?
\newline(c) A MINITAB simple linear regression print-
out is also shown (p. 100). Find the equation
of the best-fitting line through the data on
the printout.
\newline(d) Interpret the y-intercept of the least squares
line. Does it have a practical meaning for this
application? Explain.
\newline(e) Interpret the slope of the least squares line.
Over what range of x is the interpretation
meaningful?
\newline(f) Use the least squares model to estimate
the mean sale price of a property appraised
at \$300,000.


%--------------------------------------------

\begin{homeworkSection}{(a)} % Section within problem

\vspace{10pt} % Question

\problemAnswer{ % Answer

Proposed linear fit = $\hat y = 1.408x_1 + 1.359$

}
\end{homeworkSection}

%--------------------------------------------

\begin{homeworkSection}{(b)} % Section within problem
\problemAnswer{ % Answer
The linear model proposed does appear to fit the graph
}
\end{homeworkSection}

%--------------------------------------------


\begin{homeworkSection}{(c)} % Section within problem
\problemAnswer{ % Answer
Proposed linear fit = $\hat y = 1.36x_1 + 1.40827$
}
\end{homeworkSection}

%--------------------------------------------

\begin{homeworkSection}{(d)} % Section within problem
\problemAnswer{ % Answer
The y-intercept in this case represents the transaction price for a sale with the market value being zero. ie: for any transaction, an expense of \$1408.27 will be added.
}
\end{homeworkSection}

%--------------------------------------------
\begin{homeworkSection}{(e)} % Section within problem
\problemAnswer{ % Answer
The positive slope of 1.36 is interpreted as the preice relationship between the market price and the sales price of a house in the area given the range of the slopes 
}
\end{homeworkSection}

%--------------------------------------------
\begin{homeworkSection}{(f)} % Section within problem
\problemAnswer{ % Answer
y=1.36(300) + 1.40827
}
\end{homeworkSection}

%--------------------------------------------
\end{homeworkProblem}

%----------------------------------------------------------------------------------------
%	Q14
%----------------------------------------------------------------------------------------

\begin{homeworkProblem}[Question 14] % Roman numerals

Extending the life of an aluminum smelter pot. An
investigation of the properties of bricks used to line
aluminum smelter pots was published in the Amer-
ican Ceramic Society Bulletin (February 2005). Six
different commercial bricks were evaluated. The
life length of a smelter pot depends on the porosity
of the brick lining (the less porosity, the longer the
life); consequently, the researchers measured the
apparent porosity of each brick specimen, as well
as the mean pore diameter of each brick. The data
are given in the accompanying table
\newline
(a) Find the least squares line relating porosity(y) to mean pore diameter (x).
\newline
(b) Interpret the y-intercept of the line.
\newline
(c) Interpret the slope of the line.
\newline
(d) Predict the apparent porosity percentage
for a brick with a mean pore diameter of
10 micrometers.

%--------------------------------------------

\begin{homeworkSection}{(a)} % Section within problem

\vspace{10pt} % Question

\problemAnswer{ % Answer

Proposed least squares fit : $\hat y = 0.950x_1 + 6.518$

}
\end{homeworkSection}

%--------------------------------------------

\begin{homeworkSection}{(b)} % Section within problem
\problemAnswer{ % Answer
The intercept appears to represent the Porosity at which the lifespan reaches zero, or the porostiy threshold at which pots are no longer used. 
}
\end{homeworkSection}

%--------------------------------------------


\begin{homeworkSection}{(c)} % Section within problem
\problemAnswer{ % Answer
	The slope represents the ratio between the porosity of the brick meausured in small local points vs. the porosity of the entire brick.
}
\end{homeworkSection}

%--------------------------------------------

\begin{homeworkSection}{(d)} % Section within problem
\problemAnswer{ % Answer
	$\hat y = 6.3518 + 0.9498\times 10 = 15.8498$
	\newline
	A brick with the mean pore diameter of 10 micrometers is predicted to have a porosity of 15.8498\%.
}
\end{homeworkSection}


%--------------------------------------------
\end{homeworkProblem}

%Q20

\begin{homeworkProblem}[Question 20 ] % Roman numerals
Extending the life of an aluminum smelter pot.
Refer to the American Ceramic Society Bulletin
(February 2005) study of bricks that line aluminum
smelter pots, Exercise 3.14 (p. 103). You fit the sim-
ple linear regression model relating brick porosity
(y) to mean pore diameter (x) to the data in the
SMELTPOT file.
\newline\newline
(a) Find an estimate of the model standard deviation, $\sigma$.
\newline
(b) In Exercise 3.14d, you predicted brick poros-
ity percentage when x = 10 micrometers. Use
the result, part a, to estimate the error of
prediction.


%--------------------------------------------

\begin{homeworkSection}{(a)} % Section within problem
\problemAnswer{
 	\begin{equation}
		SSE = \sigma^2 = \frac{\sum_{1}^{i}(y_i - \hat{y}_i)^2}{n-2} = {\sigma}^2 \rightarrow \sigma = \sqrt{SSE}
	\end{equation}
	
	\begin{equation}
		s^2 = \frac{SSE}{d.f.} = \frac{40.551}{4} = 10.13; 	s = \sqrt(10.13) = 3.184
	\end{equation}
}
\end{homeworkSection}

%--------------------------------------------

\begin{homeworkSection}{(b)} % Section within problem
\problemAnswer{ % Answer
The estimate for a bricks porosity when the observed porosity is $10\mu m$
	\begin{equation}
		f(10\mu m)=(15.850 \pm 3.184) \%
			\end{equation}
}
\end{homeworkSection}

%--------------------------------------------

%--------------------------------------------
\end{homeworkProblem}
% Q22

\begin{homeworkProblem}[Question 22 ] % Roman numerals
Thermal characteristics of fin-tubes.
A study
was conducted to model the thermal performance
of integral-fin tubes used in the refrigeration
and process industries (Journal of Heat Transfer,
August 1990). Twenty-four specially manufac-
tured integral-fin tubes with rectangular fins made
of copper were used in the experiment. Vapor
was released downward into each tube and the
vapor-side heat transfer coefficient (based on the
outside surface area of the tube) was measured.
The dependent variable for the study is the heat
transfer enhancement ratio, y, defined as the ratio
of the vapor-side coefficient of the fin tube to
the vapor-side coefficient of a smooth tube evalu-
ated at the same temperature. Theoretically, heat
transfer will be related to the area at the top of
the tube that is ‘‘unflooded’’ by condensation of
the vapor. The data in the table are the unflooded
area ratio (x) and heat transfer enhancement (y)
values recorded for the 24 integral-fin tubes.
\newline \newline
\newline
(a) Fit a least squares line to the data.
\newline
(b) Plot the data and graph the least squares line
as a check on your calculations.
\newline
(c) Calculate SSE and $s^2$.
\newline
(d) Calculate s and interpret its value.


%--------------------------------------------

\begin{homeworkSection}{(a)} % Section within problem
\problemAnswer{
	Coefficients:\newline
(Intercept)        ratio  \newline
     0.2134       2.4264  \newline
     \newline
     $\hat y = 2.4264x_1 + 0.2134$
}
\end{homeworkSection}

%--------------------------------------------

\begin{homeworkSection}{(b)} % Section within problem
\problemAnswer{ % Answer
	\includegraphics[scale=.5]{graphs/22b}
}
\end{homeworkSection}

%--------------------------------------------


\begin{homeworkSection}{(c)} % Section within problem
\problemAnswer{ % Answer 
	SSE = 4.531 \\
	$s{2}= 0.206$
}
\end{homeworkSection}

%--------------------------------------------

\begin{homeworkSection}{(d)} % Section within problem
\problemAnswer{ % Answer
	$s= \sqrt{s^2}=0.454$ is the standard deviation of the linear model.\\
	68\% of values will fall within one $\sigma$ of the mean, and 95\% will fall within 2 
}\end{homeworkSection}
%--------------------------------------------
\end{homeworkProblem}

%Q28

\begin{homeworkProblem}[Question 28 ] % Roman numerals
Massage therapy for boxers. The British Jour-
nal of Sports Medicine (April 2000) published a study of the effect of massage on boxing perfor-
mance. Two variables measured on the boxers
were blood lactate concentration (mM) and the
boxer’s perceived recovery (28-point scale). Based
on information provided in the article, the data in
the table were obtained for 16 five-round boxing
performances, where a massage was given to the
boxer between rounds. Conduct a test to deter-
mine whether blood lactate level (y) is linearly
related to perceived recovery (x). Use $\alpha$ = .10.


%--------------------------------------------

\begin{homeworkSection}{(a)} % Section within problem
\problemAnswer{
	Based on the scatterplot, the data seems to have a weak positive trend between lactation and recovery rate. 
	\\
	\includegraphics[scale=.5]{graphs/22a}
}
\end{homeworkSection}

\begin{homeworkSection}{(b)} % Section within problem
\problemAnswer{

\begin{align*}
	H_o : \ \hat R = 0
	\\
	H_a: \ \hat R \neq  0
\end{align*}

\includegraphics[scale=.5]{graphs/22b}


Tests indicated that 
\begin{align*}
	R=0.3251
	\\
	R^2 = 0.2769
	\\
	p-value: 0.0211 \implies 
\end{align*}
The p-value is small ($p-value \leq (\alpha =0.1)$) which means that we must reject $H_0$. 
\newline
\newline
Conclusion:\\
The correlation between the two variables is non-zero and estimated to be 90\% certain that the level of correlation is between $0.189 \pm $

}

\end{homeworkSection}

\end{homeworkProblem}

%----------------------------------------------------------

%Q32

\begin{homeworkProblem}[Question 32 ] % Roman numerals
Thermal characteristics of fin-tubes. Refer to the
Journal of Heat Transfer study of the straight-line
relationship between heat transfer enhancement
(y) and unflooded area ratio (x), Exercise 3.22
(p. 109). Construct a 95\% confidence interval
\\
Data set : HEAT%-------------------------------------------

\begin{homeworkSection}{(a)} % Section within problem
\problemAnswer{
	Plotting heat on the y-axis and ratio on the x-axis, a clear, positive correlation can be inferred.\\
	Testing for $\hat R= (0.81, 0.96)$, which is reasonable enough to continure.   

		\includegraphics[scale=.45]{graphs/32a}
		\\
		\includegraphics[scale = 0.45]{graphs/Q32b-rgraph.png}

}
\end{homeworkSection}

	\begin{homeworkSection}{(b)} % Section within problem
	\problemAnswer{
	The linear model approximation yields \\
	$\hat \beta_{1}\pm (t_{\alpha}/2)s{{\hat \beta _1}}$\\
	$\hat \beta_{1} = 2.424 \pm(3.182)0.2283$\\
		95\% Confidence interval for $\hat \beta _1 : (1.698,3.150)$
	}
	\end{homeworkSection}

\end{homeworkProblem}


%---------------------------------------
%Q40 

\begin{homeworkProblem}[Question 40]
%Problem Description here
Predicting home sales price. Refer to the data on
sale prices and total appraised values of 76 residen-
tial properties recently sold in an upscale Tampa,
Florida, neighborhood, Exercise 3.8 (p. 100). The
MINITAB simple linear regression printout relat-
ing sale price (y) to appraised property (market)
value (x) is reproduced on the next page, followed
by a MINITAB correlation printout. 
\newline
(a) Find the coefficient of correlation between
appraised property value and sale price on the
printout. Interpret this value.
\newline
(b) Find the coefficient of determination between
appraised property value and sale price on the
printout. Interpret this value.

	\begin{homeworkSection}{(a)} % Section within problem
	\problemAnswer{ 

	$H_0 : \hat R  = 0 $
	\\
	$H_a : \hat R  \neq 0 $
	\\
	p-value from test: 2.2e-16.\\

P is incredibly small therefore, we must reject $H_0$


	Proposed Correlation Coefficient: 0.9755\\

	}
	\end{homeworkSection}

	\begin{homeworkSection}{(b)} % Section within problem
	\problemAnswer{
	Coefficient of Determination = $R^2$ = $0.9755^2$ = 0.9516
	}
	\end{homeworkSection}
\end{homeworkProblem}

%---------------------------------------
%Q54
\begin{homeworkProblem}[Question 54]
Recalling student names. Refer to the Journal of
Experimental Psychology—Applied (June 2000)
name retrieval study, Exercise 3.15 (p. 103).
\\
(a) Find a 99\% confidence interval for the mean
recall proportion for students in the fifth
position during the ‘‘name game.’’ Interpret
the result.
\\
(b) Find a 99\% prediction interval for the recall
proportion of a particular student in the fifth
position during the ‘‘name game.’’ Interpret
the result.
\\
(c) Compare the two intervals, parts a and b.
Which interval is wider? Will this always be
the case? Explain.
	\begin{homeworkSection}{(setup)} % Section within problem
	\problemAnswer{

		\includegraphics[scale=.5]{graphs/54a}
		\\

		a- The initial scatter plot does not seem to have any kind of meaningful correlation. 
		\\
		}
		\newpage
		
		\includegraphics[scale=.5]{graphs/54b}
				\\
		b- After constructing a linear regression, the line still does not seem to correlate.
		\\
		Testing for correlation ($\hat R$): 
		$H_0: \rho = 0$
		\\
		$H_a: \rho \neq 0$
		\\
		p-value = 0.0049, therefore we must reject $H_0$
		\\ 
		Conclusion from correlation test is that $\rho$ is not equal to 0, but after constructing a 99\% confidence interval for the value: (0.0207, 0.4256), the correlation is weak at best
		
	\end{homeworkSection}

	\begin{homeworkSection}{(a)}
	\problemAnswer{
		Confidence interval for students mean recall proportion position 5 is : $(0.453, 0.929) ; \ 0.691\pm 0.238 $
		}
	\end{homeworkSection}

	\begin{homeworkSection}{(b)}
	\problemAnswer{
		Prediction interval for students mean recall proportion in position 5 is : $(0.198, 1.207) ; \ 0.703\pm 0.501 $
		}
	\end{homeworkSection}

	\begin{homeworkSection}{(c)}
	\problemAnswer{
		The prediction interval is wider by a $2(E_2 - E_1)= 2(0.263)= 0.526$, which comes from the 1 that appears in the prediction interval and not the confidence interval \\
		\begin{equation}
			Confidence \ Interval \ = \ \hat y \pm (t_{\alpha/2})s \sqrt{\frac{1}{n} + \frac{(x_p-\bar x)^2}{SS_{xx}}}
		\end{equation}
		and \\
		\begin{equation}
			Prediction \ Interval \ = \ \hat y \pm (t_{\alpha/2})s \sqrt{1 + \frac{1}{n} + \frac{(x_p-\bar x)^2}{SS_{xx}}}
		\end{equation}
		Therefore, the prediction interval will always be larger than the confidence interval. 
	}
	\end{homeworkSection}

\end{homeworkProblem}

%---------------------------------------
%Q60 
\begin{homeworkProblem}[Question 60]
Ranking driving performance of professional
golfers. A group of Northeastern University
researchers developed a new method for rank-
ing the total driving performance of golfers on the
Professional Golf Association (PGA) tour (Sport
Journal, Winter 2007). The method requires know-
ing a golfer’s average driving distance (yards) and
driving accuracy (percent of drives that land in
the fairway). The values of these two variables
are used to compute a driving performance index.
Data for the top 40 PGA golfers (as ranked by
the new method) are saved in the PGADRIVER
file. (The first five and last five observations are
listed in the table below.) A professional golfer
is practicing a new swing to increase his average
driving distance. However, he is concerned that
his driving accuracy will be lower. Is his concern
a valid one? Use simple linear regression, where
y = driving accuracy and x = driving distance, to
answer the question.

	\begin{homeworkSection}{(a)} % Section within problem
	\problemAnswer{
%scatterplot -> explain
\includegraphics[scale=.5]{graphs/60a}
\\ The Scatter plot appears to have a clear negative correlation
%correlation/test -> explain
\\
$H_0 \ R= \ 0$ \\ 
$H_a \ R\neq \ 0 $ \\
Test results show a p-value =  8.478e-16, therefore very small, which means it is necessary to reject $H_0$. The 95\% confidence interval for this correlation is (-0.950, -0.829). With this degree of certainty in correlation, We can proceed to anaylsis. 
%linear model -> explain
}
\newpage
\problemAnswer{
Linear Regression:\\
\includegraphics[scale=.5]{graphs/60b}

Estimates: \\ 
	$\hat \beta_0 = 250.142 \pm (3.182) 14.231$ \\
	$\hat \beta_1 = -0.629 \pm (3.182)0.048$ \\ 
	$\hat R^2 = 0.8216$ \\
	$\hat R = (-0.950,-0.829)$
}
\\Assuming for Error: 
\\ (1) E($\epsilon$) = 0 
\\ (2) Var($\epsilon$) = σ 2 is constant for all x-values
\\ (3) $\epsilon$ has a normal distribution
\\ (4) $\epsilon$ ’s are independent

\begin{equation}
	s^2 = \frac{SSE}{d.f.} = \frac{874.99}{38} = 23.03
\end{equation}
s = 4.799
	\end{homeworkSection}

	\begin{homeworkSection}{b}
	\problemAnswer{
	We are 95\% confident that the following interval includes the mean decrease of accuracy per unit distance for golfing accuracies. \\
	$\hat \beta_1 = -0.629 \pm (3.182)0.048$ \\ 
	\\
	In conclusion:\\
	The golfers' concern of increasing his average driving distance, only to have his accuracy fall is a valid one. The highly correlated linear model that explains 95\% of the values depicts a model in which: 
	\\ $\hat y= (-0.629x_1 + 250.142)\%$, which corresponds to driving accuracy for a swing; accurate between 283.2	yards and 318.2 yards. For every yard that the ball travels, the golfer loses 0.629\%
	accuracy. The concern is valid.
	}
	\end{homeworkSection}
\end{homeworkProblem}


\end{document}
