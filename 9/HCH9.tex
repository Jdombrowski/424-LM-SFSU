%%%%%%%%%%%%%%%%%%%%%%%%%%%%%%%%%%%%%%%%%
% Structured General Purpose Assignment
% LaTeX Template
%
% This template has been downloaded from:
% http://www.latextemplates.com
%
% Original author:
% Ted Pavlic (http://www.tedpavlic.com)
%
% Note:
% The \lipsum[#] commands throughout this template generate dummy text
% to fill the template out. These commands should all be removed when 
% writing assignment content.
%
%%%%%%%%%%%%%%%%%%%%%%%%%%%%%%%%%%%%%%%%%

%-------------------z---------------------------------------------------------------------
%	PACKAGES AND OTHER DOCUMENT CONFIGURATIONS
%----------------------------------------------------------------------------------------

\documentclass{article}

\usepackage{fancyhdr} % Required for custom headers
\usepackage{lastpage} % Required to determine the last page for the footer
\usepackage{extramarks} % Required for headers and footers
\usepackage{graphicx} % Required to insert images

\usepackage{lipsum} % Used for inserting dummy 'Lorem ipsum' text into the template
\usepackage{amsmath}
\usepackage{verbatim}

\usepackage{listings}
\usepackage{xcolor}

\lstset{
  basicstyle=\ttfamily,
  escapeinside=||
}
% Margins
\topmargin=-0.45in{}
\evensidemargin=0in
\oddsidemargin=0in
\textwidth=6.5in
\textheight=9.0in
\headsep=0.25in 

\linespread{1.1} % Line spacing

% Set up the header and footer
\pagestyle{fancy}
\lhead{\hmwkAuthorName} % Top left header
\chead{\hmwkClass\ (\hmwkClassInstructor\ \hmwkClassTime): \hmwkTitle} % Top center header
\rhead{\firstxmark} % Top right header
\lfoot{\lastxmark} % Bottom left footer
\cfoot{} % Bottom center footer
\rfoot{Page\ \thepage\ of\ \pageref{LastPage}} % Bottom right footer
\renewcommand\headrulewidth{0.4pt} % Size of the header rule
\renewcommand\footrulewidth{0.4pt} % Size of the footer rule

\setlength\parindent{0pt} % Removes all indentation from paragraphs

%----------------------------------------------------------------------------------------
%	DOCUMENT STRUCTURE COMMANDS
%	Skip this unless you know what you're doing
%----------------------------------------------------------------------------------------

% Header and footer for when a page split occurs within a problem environment
\newcommand{\enterProblemHeader}[1]{
\nobreak\extramarks{#1}{#1 continued on next page\ldots}\nobreak
\nobreak\extramarks{#1 (continued)}{#1 continued on next page\ldots}\nobreak
}

% Header and footer for when a page split occurs between problem environments
\newcommand{\exitProblemHeader}[1]{
\nobreak\extramarks{#1 (continued)}{#1 continued on next page\ldots}\nobreak
\nobreak\extramarks{#1}{}\nobreak
}

\setcounter{secnumdepth}{0} % Removes default section numbers
\newcounter{homeworkProblemCounter} % Creates a counter to keep track of the number of problems


% PROBLEM 8
\newcommand{\homeworkProblemName}{}
\newenvironment{homeworkProblem}[1][Problem \arabic{homeworkProblemCounter}]{ % Makes a new environment called homeworkProblem which takes 1 argument (custom name) but the default is "Problem #"
\stepcounter{homeworkProblemCounter} % Increase counter for number of problems
\renewcommand{\homeworkProblemName}{#1} % Assign \homeworkProblemName the name of the problem
\section{\homeworkProblemName} % Make a section in the document with the custom problem count
\enterProblemHeader{\homeworkProblemName} % Header and footer within the environment
}{
\exitProblemHeader{\homeworkProblemName} % Header and footer after the environment
}

\newcommand{\problemAnswer}[1]{ % Defines the problem answer command with the content as the only argument
\noindent\framebox[\columnwidth][c]{\begin{minipage}{0.98\columnwidth}#1\end{minipage}} % Makes the box around the problem answer and puts the content inside
}

\newcommand{\homeworkSectionName}{}
\newenvironment{homeworkSection}[1]{ % New environment for sections within homework problems, takes 1 argument - the name of the section
\renewcommand{\homeworkSectionName}{#1} % Assign \homeworkSectionName to the name of the section from the environment argument
\subsection{\homeworkSectionName} % Make a subsection with the custom name of the subsection
\enterProblemHeader{\homeworkProblemName\ [\homeworkSectionName]} % Header and footer within the environment
}{
\enterProblemHeader{\homeworkProblemName} % Header and footer after the environment
}
   
%----------------------------------------------------------------------------------------
%	NAME AND CLASS SECTION
%----------------------------------------------------------------------------------------

\newcommand{\hmwkTitle}{Homework Chapter 9: 
\\Logistic Regression} % Assignment title
\newcommand{\hmwkDueDate}{Wednesday,\ December\ 6,\ 2017} % Due date
\newcommand{\hmwkClass}{MATH\ 424} % Course/class
\newcommand{\hmwkClassTime}{11:10am} % Class/lecture time
\newcommand{\hmwkClassInstructor}{Kafai} % Teacher/lecturer
\newcommand{\hmwkAuthorName}{Jonathan Dombrowski} % Your name

%----------------------------------------------------------------------------------------
%	TITLE PAGE
%----------------------------------------------------------------------------------------

\title{
\vspace{2in}
\textmd{\textbf{\hmwkClass:\ \hmwkTitle}}\\
\normalsize\vspace{0.1in}\small{Due\ on\ \hmwkDueDate}\\
\vspace{0.1in}\large{\textit{\hmwkClassInstructor\ \hmwkClassTime}}
\vspace{3in}
}

\author{\textbf{\hmwkAuthorName}}
\date{} % Insert date here if you want it to appear below your name

%----------------------------------------------------------------------------------------

\begin{document}

\maketitle

%----------------------------------------------------------------------------------------
%	TABLE OF CONTENTS
%----------------------------------------------------------------------------------------

%\setcounter{tocdepth}{1} % Uncomment this line if you don't want subsections listed in the ToC

\newpage
\tableofcontents
\newpage

%----------------------------------------------------------------------------------------

%---------------------------------------------
% Q 24
%---------------------------------------------
\begin{homeworkProblem}[Q 24]
%Question material goes header
Flight response of geese. Offshore oil drilling near
an Alaskan estuary has led to increased air traf-
fic—mostly large helicopters—in the area. The
U.S. Fish and Wildlife Service commissioned a
study to investigate the impact these helicopters
have on the flocks of Pacific brant geese, which
inhabit the estuary in Fall before migrating (Sta-
tistical Case Studies: A Collaboration between
Academe and Industry, 1998). Two large heli-
copters were flown repeatedly over the estuary
at different altitudes and lateral distances from the
flock. The flight responses of the geese (recorded as ‘‘low’’ or ‘‘high’’), altitude (x 1 = hundreds of
meters), and lateral distance (x 2 = hundreds of
meters) for each of 464 helicopter overflights were
recorded and are saved in the PACGEESE file.
(The data for the first 10 overflights are shown in
the table, p. 503.) MINITAB was used to fit the
logistic regression model π ∗ = β 0 + β 1 x 1 + β 2 x 2 ,
where y = {1 if high response, 0 if low response},
π = P(y = 1), and π ∗ = ln[π/(1 − π )]. The result-
ing printout is shown above.
\\
(a) Is the overall logit model statistically useful for
predicting geese flight response? Test using
α = .01.
\\
(b) Conduct a test to determine if flight response
of the geese depends on altitude of the heli-
copter. Test using α = .01.
\\
(c) Conduct a test to determine if flight response
of the geese depends on lateral distance of
helicopter from the flock. Test using α = .01.
\\
(d) Predict the probability of a high flight response
from the geese for a helicopter flying over the
estuary at an altitude of x 1 = 6 hundred meters
and at a lateral distance of x 2 = 3 hundred
meters.

\begin{homeworkSection}{a}
	\problemAnswer{
	Using the model output in the text as reference:
		\[
			H_0 = \hat\beta_1 = \hat\beta_2 = 0
		\]
		\[
			H_a = \hat\beta_1, \hat\beta_2 \neq 0
		\]
		We can compare the Chi squared values and their respective d.f.'s by invoking \\
		`1-pchisq(model$null.deviance-model$deviance, model$df.null-model$df.residual)`\\
		In the same motion as the F-test as we would do for a linear regression: taking the ratio of two Chi squared values and exracting a p-value from that. 
		\\ 
		We can extract a p-value = 0. p is less than $\alpha =0.01$, therefore we can reject $H_{0}$ and state that the model is statistically useful for predicting geese flight response. 
	}
\end{homeworkSection}

\begin{homeworkSection}{b}
	\problemAnswer{
		\[
			H_0 = \hat\beta_1 = 0
		\]
		\\
		\[
			H_a = \hat\beta_1 \neq 0
		\]
		by looking at the model summary, we can see that the z-score for this test is z=2.914 , with a corresponding p-value of 0.00357. We can conclude that p is less than $\alpha$ and therefore that the flight response does in fact depend on the altitude of the helicopter. 
	}
\end{homeworkSection}

\begin{homeworkSection}{c}
	\problemAnswer{
		\[
			H_0 = \hat\beta_2 = 0
		\]
		\[
			H_a = \hat\beta_2 \neq 0
		\]
		by looking at the model summary, we can see that the z-score for this test is z=-10.625 , with a corresponding p-value of less than $2\times10^{-16}$. We can conclude that p is less than $\alpha=0.01$ and therefore that the flight response does in fact depend on the lateral distance of the helicopter. 
	}

	\begin{homeworkSection}{d}
		\problemAnswer{
			For \[{{x_1 = 6, x_2 = 3}}\] where both are measured in hundreds of meters, the predicted probability from the logistic model 
			\[
				\pi^* =  \beta_0 + \beta_1x_1 + \beta_2x_2 
			\]
			using the R commands:\\
			: newdata = data.frame(ALTITUDE=6, LATERAL=3)\\
			: predict(model,newdata,type="response")\\
			is 0.946. From this we can conclude that there is a 94.6\% chance that the geese have a high flight response when the Altitude is held at a fixed 600m, and the Lateral distance is held at a fixed 300m based on the data reported.
		}
	\end{homeworkSection}
		
\end{homeworkSection}
	
\end{homeworkProblem}

%---------------------------------------------
% Q 26
%---------------------------------------------
\begin{homeworkProblem}[Q 26]
%Question material goes here
Groundwater contamination in wells. Many New
Hampshire counties mandate the use of reformulated gasoline, leading to an increase in groundwater contamination. Refer to the Environmental
Science and Technology (January 2005) study
of the factors related to methyl tert-butyl ether
(MTBE) contamination in public and private
New Hampshire wells, Exercise 6.11 (p. 343). Data
were collected for a sample of 223 wells and
are saved in the MTBE file. Recall that the list
of potential predictors of MTBE level include
well class (public or private), aquifer (bedrock or
unconsolidated), pH level (standard units), well
depth (meters), amount of dissolved oxygen (mil
ligrams per liter), distance from well to nearest fuel
source (meters), and percentage of adjacent land
allocated to industry. For this exercise, consider the
dependent variable y = {1 if a ‘‘detectible level’’
of MTBE is found, 0 if the level of MTBE found
is ‘‘below limit’’}. Using the independent variables
indentified in Exercise 6.11, fit a logistic regres
sion model for the probability of a ‘‘detectible
level’’ of MTBE. Interpret the results of the logistic
regression. Do you recommend using the model?
Explain.

\begin{homeworkSection}{a}
The model in question is defined as follows: 
\\
\[\pi^{*}=
\beta_0 + \beta_{1}x_1 + \beta_{2}x_2 + \beta_{3}x_{3}+\beta_{4}x_{4}+\beta_{5}x_{5}+\beta_{6}x_{6}+\beta_{7}x_{7}\]
With \\
\[ 
x_1
 \bigg\{
  \begin{tabular}{cc}
  0 :  \text{Below limit}\\
  1 :  \text{Detectable}
  \end{tabular}
\]

\begin{lstlisting}
	Call:
glm(formula = detet ~ wellclass + aquifer + ph + welldepth + 
    do2 + d2f + pind, family = binomial(link = "logit"))

Deviance Residuals: 
    Min       1Q   Median       3Q      Max  
-1.4664  -0.8663  -0.6549   1.0395   2.2075  

Coefficients:
                   Estimate Std. Error z value Pr(>|z|)  
(Intercept)       1.1713423  1.7544128   0.668   0.5044  
wellclassPublic   0.8066518  0.3807905   2.118   0.0341 *
aquiferUnconsoli -0.2693525  0.6760558  -0.398   0.6903  
ph               -0.4098715  0.2328449  -1.760   0.0784 .
welldepth         0.0084778  0.0034056   2.489   0.0128 *
do2               0.0062126  0.0742396   0.084   0.9333  
d2f              -0.0001209  0.0001642  -0.736   0.4617  
pind              0.0234640  0.0317899   0.738   0.4605  
---
Signif. codes:  0 ‘***’ 0.001 ‘**’ 0.01 ‘*’ 0.05 ‘.’ 0.1 ‘ ’ 1

(Dispersion parameter for binomial family taken to be 1)

    Null deviance: 242.21  on 190  degrees of freedom
Residual deviance: 221.13  on 183  degrees of freedom
  (32 observations deleted due to missingness)
AIC: 237.13

Number of Fisher Scoring iterations: 4

	\end{lstlisting}


\includegraphics[scale=0.2]{graphs/26a.png}
\\
Looking at the graphs for this model from the R output, we can see that the residuals for both groups are distributed normally, with none of them breaching 3 standards deviations, or the Cooks' distance. So the weight of any single point does not overly pull the model in any direction too far. 
\\
Continuing the analysis, we test the model for overall model adequacy. 
\\
\[
	H_0 = \hat\beta_1 = \hat\beta_2 = \ldots = \hat\beta_7 = 0
\]
\[
	H_a = \text{at least one }\beta_i(i=0,7) \neq 0
\]
\\
To do this we use the same method as in Question 24. From our $\chi^{2}$ testing, we get a p-value = 0.00365, which: p-value  is less than $\alpha=0.5$, so we can reject the null hypothesis and state that the model is statistically useful for predicting the probability of detection in MTBE levels. 
\\
After globally validating the model, we can then move to see if each of the $\hat\beta$'s are statistically significant. 
\\
For i = 1 to 7 :
\\
\[
	H_0 : \hat\beta_i = 0
\]
\[
	H_a : \hat\beta_i \neq 0
\]
\\
We can refer to the R printout to see the p-values associated with the z-scores for the individual betas. After doing a backwards variable selection based on the p-values, we arrive at a model predicting the Detection status of MTBE with the WellClass, pH level, and WellDepth. The p-values for the respective predictors are 0.00575, 0.05476, 0.00187. The pH level was the only one to come close to the default $\alpha = 0.05$. Testing overall model adequacy again, we use the $\chi^{2}$ testing method form Q 24, we arrive at a p-value of 0.000258. Comparing this to the previous model's p-value of 0.00365. Following the manual variable selection, we have improved the p-value by a factor of 10.
\\
\begin{lstlisting}
	Call:
glm(formula = detet ~ wellclass + ph + welldepth, family = binomial(link = "logit"))

Deviance Residuals: 
    Min       1Q   Median       3Q      Max  
-1.4267  -0.8812  -0.6763   1.0872   2.1673  

Coefficients:
                  Estimate Std. Error z value Pr(>|z|)   
(Intercept)       0.834621   1.535032   0.544  0.58664   
wellclassPublic   0.958560   0.347112   2.762  0.00575 **
ph               -0.410929   0.213941  -1.921  0.05476 . 
welldepth         0.009641   0.003100   3.111  0.00187 **
---
Signif. codes:  
0 ‘***’ 0.001 ‘**’ 0.01 ‘*’ 0.05 ‘.’ 0.1 ‘ ’ 1

(Dispersion parameter for binomial family taken to be 1)

    Null deviance: 242.21  on 190  degrees of freedom
Residual deviance: 223.09  on 187  degrees of freedom
  (32 observations deleted due to missingness)
AIC: 231.09

Number of Fisher Scoring iterations: 4
\end{lstlisting}

\includegraphics[scale=0.25]{graphs/26b.png}
\\
Looking at the residuals quickly, after model reformation, we see that nothing has changed in respect to the status of normality or leverage. 
\\
Depending on the tolerance of the client requesting the model, I would also reccommend removing the pH term as the p-value was just above 0.05. Depending on the desire of the company, I would reccommend either this model, or this model with the pH value removed.  
\\
Interpretation of Betas
\\
The betas in their current states represent changes in the log-odds $\pi^{*}$. In order to obtain the percentage change in odds for each unit increase, we must first transform it back to find the percentage change in odds. We calculate for all i's (1,3)
\[
	e^{\hat\beta_i}-1	
\] 
\\%0.347
Interpretation of $\hat\beta_{1}$: this corresponds to the the well being public or not. If the well is public, the well has a 160\% higher odds of having detectable levels of MTBE while keeping any other x's held fixed.
\\
%0.411
Interpretation of $\hat\beta_{2}$: this corresponds to the the wells' pH value. For every one unit increase in the pH value of the well, the odds of the well having detectable levels of MTBE decreases by 33.7\% while keeping any other x's held fixed.
\\
Interpretation of $\hat\beta_{3}$: this corresponds to the the wells' depth. For every one unit increase in the depth of the well in meters, the odds that the well will have detectable levels of MTBE will decrease by 0.959\% while keeping any other x's held fixed.
\\
The model proposed estimates the probability of the level of the MTBE level being low or high. With 0 being ``Below Limit'', and 1 being ``Detect''. The higher the result of the model, the more likely that the well has a detectable level of MTBE.

\end{homeworkSection}

\end{homeworkProblem}

%---------------------------------------------
% Q 28%---------------------------------------------
\begin{homeworkProblem}[Q 28]
A new dental bonding agent. When bonding
teeth, orthodontists must maintain a dry field. A
new bonding adhesive (called ‘‘Smartbond’’) has
been developed to eliminate the necessity of a
dry field. However, there is concern that the new
bonding adhesive may not stick to the tooth as
well as the current standard, a composite adhesive
(Trends in Biomaterials and Artificial Organs, Jan-
uary 2003). Tests were conducted on a sample of
10 extracted teeth bonded with the new adhesive
and a sample of 10 extracted teeth bonded with
the composite adhesive. The Adhesive Remnant
Index (ARI), which measures the residual adhe-
sive of a bonded tooth on a scale of 1 to 5, was
determined for each of the 20 bonded teeth after 1
hour of drying. (Note: An ARI score of 1 implies
all adhesive remains on the tooth, while a score
of 5 means none of the adhesive remains on the
tooth.) The data are listed in the accompanying
table. Fit a logistic regression model for the prob-
ability of the new (Smartbond) adhesive based on
the ARI value of the bonded tooth. Interpret the
results.
\begin{homeworkSection}{a}
Fitting the model, \\
\[
	\pi^* = \beta_0 + \beta_1x_1
\]
\[ 
x_1
\bigg\{
  \begin{tabular}{cc}
  0 :  \text{Composite}\\
  1 :  \text{SmartBond}
  \end{tabular}
\]
Fitting model to the data gives us the following R output table:
\begin{lstlisting}
Call:
glm(formula = BONDING$ADHESIVE ~ BONDING$ARISCORE, family = binomial(link = "logit"))

Deviance Residuals: 
    Min       1Q   Median       3Q      Max  
-1.9842  -1.2267   0.1391   1.1289   1.1289  

Coefficients:
                 Estimate Std. Error z value Pr(>|z|)
(Intercept)         3.521      2.179   1.616    0.106
BONDING$ARISCORE   -1.703      1.044  -1.631    0.103

(Dispersion parameter for binomial family taken to be 1)

    Null deviance: 27.726  on 19  degrees of freedom
Residual deviance: 23.447  on 18  degrees of freedom
AIC: 27.447

Number of Fisher Scoring iterations: 4
\end{lstlisting}

Taking a look at the model residuals: 
\\
\includegraphics[scale=0.25]{graphs/28a.png}
\\
There appears to be one point (11) which is close to breaching the 3 Std Deviation threshold for leverage, but other than that, nothing seems to be extraneous considering the small sample size. 
\\ 
Moving to global model validation, we look at the p-score from the $\chi ^{2}$ testing.
\\
\[
	H_0 = \beta_1 = 0
\]
\[
	H_a = \beta_1 \neq 0
\]

 The result is a p-value = 0.0386. Comparing this with a default $\alpha = 0.05$ leads to the conclusion that p , $\alpha$, therefore we can reject the null hypothesis and state that the model is statistically significant for predicting the category of adhesive used based on the ARISCORE value. 
\[
	H_0 = \hat\beta_1 = 0
\]
\[
	H_a = \hat\beta_1 \neq 0
\]
Moving to individual beta analysis, we can refer to the R printout above to determine whether or not the term itself is statistically significant. The z-score found is -1.63, with an accompanying p-value of 0.103. Therefore we can reject the null hypothesis and state that the individual beta of the ARISCORE is statistically significant in predicting the probability of the type of adhesive used on a tooth. 
\\ 
Armed with this information, we can recommend the given model in predicting the probability of the type of adhesive used on a tooth, with 1 being representative of SmartBond, and zero being representative of a Compound adhesive. 
\\
The betas in their current states represent changes in the log-odds $\pi^{*}$. In order to obtain the percentage change in odds for each unit increase, we must first transform it back to find the percentage change in odds. We calculate this for $\hat\beta_{1}$:
\[
	e^{-1.703}-1 = -0.818	
\]
To interpret $\hat\beta_{1}$, it represents the -81.8\% change in odds for each unit change in ARISCORE while keeping any other variables held fixed. 

\end{homeworkSection}

\end{homeworkProblem}
\end{document}
