%%%%%%%%%%%%%%%%%%%%%%%%%%%%%%%%%%%%%%%%%
% Structured General Purpose Assignment
% LaTeX Template
%
% This template has been downloaded from:
% http://www.latextemplates.com
%
% Original author:
% Ted Pavlic (http://www.tedpavlic.com)
%
% Note:
% The \lipsum[#] commands throughout this template generate dummy text
% to fill the template out. These commands should all be removed when 
% writing assignment content.
%
%%%%%%%%%%%%%%%%%%%%%%%%%%%%%%%%%%%%%%%%%

%-------------------z---------------------------------------------------------------------
%	PACKAGES AND OTHER DOCUMENT CONFIGURATIONS
%----------------------------------------------------------------------------------------

\documentclass{article}

\usepackage{fancyhdr} % Required for custom headers
\usepackage{lastpage} % Required to determine the last page for the footer
\usepackage{extramarks} % Required for headers and footers
\usepackage{graphicx} % Required to insert images

\usepackage{lipsum} % Used for inserting dummy 'Lorem ipsum' text into the template
\usepackage{amsmath}
\usepackage{verbatim}

\usepackage{listings}
\usepackage{xcolor}

\lstset{
  basicstyle=\ttfamily,
  escapeinside=||
}
% Margins
\topmargin=-0.45in{}
\evensidemargin=0in
\oddsidemargin=0in
\textwidth=6.5in
\textheight=9.0in
\headsep=0.25in 

\linespread{1.1} % Line spacing

% Set up the header and footer
\pagestyle{fancy}
\lhead{\hmwkAuthorName} % Top left header
\chead{\hmwkClass\ (\hmwkClassInstructor\ \hmwkClassTime): \hmwkTitle} % Top center header
\rhead{\firstxmark} % Top right header
\lfoot{\lastxmark} % Bottom left footer
\cfoot{} % Bottom center footer
\rfoot{Page\ \thepage\ of\ \pageref{LastPage}} % Bottom right footer
\renewcommand\headrulewidth{0.4pt} % Size of the header rule
\renewcommand\footrulewidth{0.4pt} % Size of the footer rule

\setlength\parindent{0pt} % Removes all indentation from paragraphs

%----------------------------------------------------------------------------------------
%	DOCUMENT STRUCTURE COMMANDS
%	Skip this unless you know what you're doing
%----------------------------------------------------------------------------------------

% Header and footer for when a page split occurs within a problem environment
\newcommand{\enterProblemHeader}[1]{
\nobreak\extramarks{#1}{#1 continued on next page\ldots}\nobreak
\nobreak\extramarks{#1 (continued)}{#1 continued on next page\ldots}\nobreak
}

% Header and footer for when a page split occurs between problem environments
\newcommand{\exitProblemHeader}[1]{
\nobreak\extramarks{#1 (continued)}{#1 continued on next page\ldots}\nobreak
\nobreak\extramarks{#1}{}\nobreak
}

\setcounter{secnumdepth}{0} % Removes default section numbers
\newcounter{homeworkProblemCounter} % Creates a counter to keep track of the number of problems


% PROBLEM 8
\newcommand{\homeworkProblemName}{}
\newenvironment{homeworkProblem}[1][Problem \arabic{homeworkProblemCounter}]{ % Makes a new environment called homeworkProblem which takes 1 argument (custom name) but the default is "Problem #"
\stepcounter{homeworkProblemCounter} % Increase counter for number of problems
\renewcommand{\homeworkProblemName}{#1} % Assign \homeworkProblemName the name of the problem
\section{\homeworkProblemName} % Make a section in the document with the custom problem count
\enterProblemHeader{\homeworkProblemName} % Header and footer within the environment
}{
\exitProblemHeader{\homeworkProblemName} % Header and footer after the environment
}

\newcommand{\problemAnswer}[1]{ % Defines the problem answer command with the content as the only argument
\noindent\framebox[\columnwidth][c]{\begin{minipage}{0.98\columnwidth}#1\end{minipage}} % Makes the box around the problem answer and puts the content inside
}

\newcommand{\homeworkSectionName}{}
\newenvironment{homeworkSection}[1]{ % New environment for sections within homework problems, takes 1 argument - the name of the section
\renewcommand{\homeworkSectionName}{#1} % Assign \homeworkSectionName to the name of the section from the environment argument
\subsection{\homeworkSectionName} % Make a subsection with the custom name of the subsection
\enterProblemHeader{\homeworkProblemName\ [\homeworkSectionName]} % Header and footer within the environment
}{
\enterProblemHeader{\homeworkProblemName} % Header and footer after the environment
}
   
%----------------------------------------------------------------------------------------
%	NAME AND CLASS SECTION
%----------------------------------------------------------------------------------------

\newcommand{\hmwkTitle}{Final Homework } % Assignment title
\newcommand{\hmwkDueDate}{Wednesday,\ December\ 20,\ 2017} % Due date
\newcommand{\hmwkClass}{MATH\ 424} % Course/class
\newcommand{\hmwkClassTime}{11:10am} % Class/lecture time
\newcommand{\hmwkClassInstructor}{Kafai} % Teacher/lecturer
\newcommand{\hmwkAuthorName}{Jonathan Dombrowski} % Your name

%----------------------------------------------------------------------------------------
%	TITLE PAGE
%----------------------------------------------------------------------------------------

\title{
\vspace{2in}
\textmd{\textbf{\hmwkClass:\ \hmwkTitle}}\\
\normalsize\vspace{0.1in}\small{Due\ on\ \hmwkDueDate}\\
\vspace{0.1in}\large{\textit{\hmwkClassInstructor\ \hmwkClassTime}}
\vspace{3in}
}

\author{\textbf{\hmwkAuthorName}}
\date{} % Insert date here if you want it to appear below your name

%----------------------------------------------------------------------------------------

\begin{document}

\maketitle

%----------------------------------------------------------------------------------------
%	TABLE OF CONTENTS
%----------------------------------------------------------------------------------------

%\setcounter{tocdepth}{1} % Uncomment this line if you don't want subsections listed in the ToC

\newpage
\tableofcontents
\newpage

%----------------------------------------------------------------------------------------
Q1. Consider the data in q1data.txt. All subjects are asthmatics. For the model with Forced Expiratory Volume (FEV) as the response and Height, Weight, and Age as the predictors,
\\
a) Examine a plot of the studentized or jackknife residuals versus the predicted values. Are any regression assumption violations apparent? If so, suggest possible remedies.
\\
b) Examine numerical descriptive statistics, histograms, box-and-whisker plots, and normal probability plots of jackknife residuals. Is the normality assumption violated? If so, suggest possible remedies.
\\
c) Examine outlier diagnostics, including Cook's distance, leverage statistics, and jackknife residuals, and identify any potential outlier’s. What course of action, if any, should be taken when outliers are identified?
d) Examine variance inflation factors, condition i.Ildices (unadjusted and adjusted for the intercept), and variance proportions. Are there any important collinearity problems? If so, suggest possible remedies.
\\
Q2. A random sample of data was collected on residential sales in a large city. The data in   q2data.txt shows the selling price (Y, in \$1,000s), area (x1, in hundreds of square feet), number of bedrooms (X2),   total number of rooms (X3),   house  age (X4,  in years), and location (Z = 0 for in-town and inner suburbs, Z=1for outer suburbs). In parts a through c, use variables X1, X2, X3, X4, and Z as the predictor variables.
\\
a) Use the all possible regressions procedure to suggest a best model.
\\
b) Use the stepwise regression algorithm to suggest a best model.
\\
c) Use the backward elimination algorithm to suggest a best model.
\\
d) Which of the models selected in a, b, and c seems to be the best model, and why?
\\
Q3. The data listed in q3data.txt relate to a study by Reiter and others concerning the effects of injecting triethyl-tin (TET) into rats once at age 5 days. The animals were injected with 0, 3, or 6 mg per kilogram of body weight. The response was the log of the activity count, log (ac), for 1 hour, recorded at 21 days of age. The rat was left to move about freely in a figure 8 maze. Analysis of other studies with this type of activity count confirms that log counts should yield Gaussian errors if the model is correct.
\\
a) Conduct a two-way ANOVA with SEX and DOSAGE as factors.
\\
b) Using α = .05, report your conclusions based on the ANOVA.
\\
c) Which, if any, families of means should be followed up with multiple-comparison tests? What type of comparisons would you recommend?
\\
Q4. The data in q4data.txt is the record of coronary artery disease (ca, 0=no, 1= yes), age, ECG (0, 1, and 2 based on the reading of ST segment depression), and sex (0=male, 1=female). Based on this model
\\
a) What is the estimated logistic regression model for the relationship between ca and age, ECG, sex?
\\
b) What is a 30-year-old male, ECG=2 predicted probability of having coronary artery disease?
\\
c) Estimate the odds ratio comparing a 30-year-old male, ECG=2 to a 31-year-old male, ECG=2. Interpret this estimated odds ratio.
\\
d) Find a 95\% confidence interval for the population odds ratio being estimated in part (c).

\end{document}
	