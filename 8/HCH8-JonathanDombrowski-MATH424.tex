%%%%%%%%%%%%%%%%%%%%%%%%%%%%%%%%%%%%%%%%%
% Structured General Purpose Assignment
% LaTeX Template
%
% This template has been downloaded from:
% http://www.latextemplates.com
%
% Original author:
% Ted Pavlic (http://www.tedpavlic.com)
%
% Note:
% The \lipsum[#] commands throughout this template generate dummy text
% to fill the template out. These commands should all be removed when 
% writing assignment content.
%
%%%%%%%%%%%%%%%%%%%%%%%%%%%%%%%%%%%%%%%%%

%-------------------z---------------------------------------------------------------------
%	PACKAGES AND OTHER DOCUMENT CONFIGURATIONS
%----------------------------------------------------------------------------------------

\documentclass{article}

\usepackage{fancyhdr} % Required for custom headers
\usepackage{lastpage} % Required to determine the last page for the footer
\usepackage{extramarks} % Required for headers and footers
\usepackage{graphicx} % Required to insert images

\usepackage{lipsum} % Used for inserting dummy 'Lorem ipsum' text into the template
\usepackage{amsmath}
\usepackage{verbatim}

\usepackage{listings}
\usepackage{xcolor}

\lstset{
  basicstyle=\ttfamily,
  escapeinside=||
}
% Margins
\topmargin=-0.45in{}
\evensidemargin=0in
\oddsidemargin=0in
\textwidth=6.5in
\textheight=9.0in
\headsep=0.25in 

\linespread{1.1} % Line spacing

% Set up the header and footer
\pagestyle{fancy}
\lhead{\hmwkAuthorName} % Top left header
\chead{\hmwkClass\ (\hmwkClassInstructor\ \hmwkClassTime): \hmwkTitle} % Top center header
\rhead{\firstxmark} % Top right header
\lfoot{\lastxmark} % Bottom left footer
\cfoot{} % Bottom center footer
\rfoot{Page\ \thepage\ of\ \pageref{LastPage}} % Bottom right footer
\renewcommand\headrulewidth{0.4pt} % Size of the header rule
\renewcommand\footrulewidth{0.4pt} % Size of the footer rule

\setlength\parindent{0pt} % Removes all indentation from paragraphs

%----------------------------------------------------------------------------------------
%	DOCUMENT STRUCTURE COMMANDS
%	Skip this unless you know what you're doing
%----------------------------------------------------------------------------------------

% Header and footer for when a page split occurs within a problem environment
\newcommand{\enterProblemHeader}[1]{
\nobreak\extramarks{#1}{#1 continued on next page\ldots}\nobreak
\nobreak\extramarks{#1 (continued)}{#1 continued on next page\ldots}\nobreak
}

% Header and footer for when a page split occurs between problem environments
\newcommand{\exitProblemHeader}[1]{
\nobreak\extramarks{#1 (continued)}{#1 continued on next page\ldots}\nobreak
\nobreak\extramarks{#1}{}\nobreak
}

\setcounter{secnumdepth}{0} % Removes default section numbers
\newcounter{homeworkProblemCounter} % Creates a counter to keep track of the number of problems


% PROBLEM 8
\newcommand{\homeworkProblemName}{}
\newenvironment{homeworkProblem}[1][Problem \arabic{homeworkProblemCounter}]{ % Makes a new environment called homeworkProblem which takes 1 argument (custom name) but the default is "Problem #"
\stepcounter{homeworkProblemCounter} % Increase counter for number of problems
\renewcommand{\homeworkProblemName}{#1} % Assign \homeworkProblemName the name of the problem
\section{\homeworkProblemName} % Make a section in the document with the custom problem count
\enterProblemHeader{\homeworkProblemName} % Header and footer within the environment
}{
\exitProblemHeader{\homeworkProblemName} % Header and footer after the environment
}

\newcommand{\problemAnswer}[1]{ % Defines the problem answer command with the content as the only argument
\noindent\framebox[\columnwidth][c]{\begin{minipage}{0.98\columnwidth}#1\end{minipage}} % Makes the box around the problem answer and puts the content inside
}

\newcommand{\homeworkSectionName}{}
\newenvironment{homeworkSection}[1]{ % New environment for sections within homework problems, takes 1 argument - the name of the section
\renewcommand{\homeworkSectionName}{#1} % Assign \homeworkSectionName to the name of the section from the environment argument
\subsection{\homeworkSectionName} % Make a subsection with the custom name of the subsection
\enterProblemHeader{\homeworkProblemName\ [\homeworkSectionName]} % Header and footer within the environment
}{
\enterProblemHeader{\homeworkProblemName} % Header and footer after the environment
}
   
%----------------------------------------------------------------------------------------
%	NAME AND CLASS SECTION
%----------------------------------------------------------------------------------------

\newcommand{\hmwkTitle}{Homework Chapter 7: 
\\Residual Analysis} % Assignment title
\newcommand{\hmwkDueDate}{Saturday,\ November\ 18,\ 2017} % Due date
\newcommand{\hmwkClass}{MATH\ 424} % Course/class
\newcommand{\hmwkClassTime}{11:10am} % Class/lecture time
\newcommand{\hmwkClassInstructor}{Kafai} % Teacher/lecturer
\newcommand{\hmwkAuthorName}{Jonathan Dombrowski} % Your name

%----------------------------------------------------------------------------------------
%	TITLE PAGE
%----------------------------------------------------------------------------------------

\title{
\vspace{2in}
\textmd{\textbf{\hmwkClass:\ \hmwkTitle}}\\
\normalsize\vspace{0.1in}\small{Due\ on\ \hmwkDueDate}\\
\vspace{0.1in}\large{\textit{\hmwkClassInstructor\ \hmwkClassTime}}
\vspace{3in}
}

\author{\textbf{\hmwkAuthorName}}
\date{} % Insert date here if you want it to appear below your name

%----------------------------------------------------------------------------------------

\begin{document}

\maketitle

%----------------------------------------------------------------------------------------
%	TABLE OF CONTENTS
%----------------------------------------------------------------------------------------

%\setcounter{tocdepth}{1} % Uncomment this line if you don't want subsections listed in the ToC

\newpage
\tableofcontents
\newpage

%----------------------------------------------------------------------------------------

%---------------------------------------------
% Q 4
%---------------------------------------------

%---------------------------------------------
% Q _
%---------------------------------------------
\begin{homeworkProblem}[Q 4]
Elasticity of moissanite. Moissanite is a popular
abrasive material because of its extreme hard-
ness. Another important property of moissanite
is elasticity. The elastic properties of the material
were investigated in the Journal of Applied Physics
(September 1993). A diamond anvil cell was used to
compress a mixture of moissanite, sodium chloride,
and gold in a ratio of 33:99:1 by volume. The com-
pressed volume, y, of the mixture (relative to the
zero-pressure volume) was measured at each of 11
different pressures (GPa). The results are displayed
in the table (p. 397). A MINITAB printout for the
straight-line regression model E(y) = β 0 + β 1 x and
a MINITAB residual plot are displayed at left.

(a) Calculate the regression residuals.
\\(b) Plot the residuals against x. Do you detect
a trend?
\\(c) Propose an alternative model based on the plot,
part b.
\\(d) Fit and analyze the model you proposed in
part c.
%Question material goes here

\begin{homeworkSection}{a}

	%Question answer goes here
	The residuals for the given model are as follows:
	\begin{lstlisting}
         1          2          3          4          5          6          7 
-161.49037  -77.65049  -35.77246  176.28837 -117.29612   53.68403  141.98597 
         8          9         10         11         12         13         14 
  50.66572  116.91583   71.68961   17.19792  -73.17866   52.86881  -34.07440 
        15         16         17         18         19         20         21 
-117.08463  210.10916  100.71196 -117.45702  213.49551  176.15586 -153.66866 
        22         23         24         25         26         27         28 
  22.62320 -185.37244  -33.27300  176.25143   76.81172 -189.54715 -138.15989 
        29         30         31         32 
 108.07673   16.11443 -206.48496 -141.13600 
	\end{lstlisting}

\end{homeworkSection}

\begin{homeworkSection}{b}
	\problemAnswer{
	\includegraphics[scale= 0.5]{graphs/4a.png}
	\\
		After plotting the residuals of the model, the plot of Fitted values vs Residuals, we can see that there is a trend in the residuals and instead of being linear, appears to be quadratic. By changing 	 	the model to a quadratic one, we can hopefully quell some of these errors. 
	}
\end{homeworkSection}

\begin{homeworkSection}{c}
	\problemAnswer{
	The new proposed model is $E(y)=\hat\beta_0 + \hat\beta_1x_1 + \hat\beta_2x_1^2$
	A second degree model will hopefully deal with the likely quadratic residual plot, and normalize the residual.
	}
\end{homeworkSection}

\begin{homeworkSection}{d}
	\includegraphics[scale = 0.5]{graphs/4b.png}
	\\
	Which we can see that the residuals are within 2s and appear to be far more randomnly distributed than the purely linear model. We can then conclude that the second order model fits the data more accurately. 

\end{homeworkSection}


\end{homeworkProblem}

%---------------------------------------------
% Q _
%---------------------------------------------
\begin{homeworkProblem}[Q 11]
%Question material goes here
E(y) = β 0 + β 1 x 1 + β 2 x 2 + β 3 x 1 x 2 .\\
Fit the model to the data saved in the GASTURBINE file, then plot the residuals against predicted heat rate. Is the assumption of a constant error variance reasonably satisfied? If not, suggest how to modify the model.

\begin{homeworkSection}{a}
Fitting the model to the data:\\
\begin{lstlisting}
lm(formula = heatrate ~ cpratio + rpm + I(cpratio * rpm))

Residuals:
    Min      1Q  Median      3Q     Max 
-1211.7  -375.6  -107.2   189.7  2095.0 

Coefficients:
                   Estimate Std. Error t value Pr(>|t|)    
(Intercept)       1.207e+04  4.185e+02  28.828  < 2e-16 ***
cpratio          -1.461e+02  2.666e+01  -5.479 7.98e-07 ***
rpm               1.697e-01  3.467e-02   4.895 7.16e-06 ***
I(cpratio * rpm) -2.425e-03  3.120e-03  -0.777     0.44    
---
Signif. codes:  0 ‘***’ 0.001 ‘**’ 0.01 ‘*’ 0.05 ‘.’ 0.1 ‘ ’ 1

Residual standard error: 633.8 on 63 degrees of freedom
Multiple R-squared:  0.8492,	Adjusted R-squared:  0.8421 
F-statistic: 118.3 on 3 and 63 DF,  p-value: < 2.2e-16
\end{lstlisting}

\includegraphics[scale = 0.5]{graphs/11a.png}

The residuals seem to be randomly distributed, possibly binomially distrubuted if anything. The residuals are not the issue with this model, instead we propose that since the t-score of the interaction term is only -0.777, with a p-value of 0.44, that it is not statistically significant. Changing to a new first degree model is what I suggest. Testing that suggestion:
\\
${{\hat y=\hat\beta_0+\hat\beta_1x_1}}$
\\
\begin{lstlisting}
lm(formula = heatrate ~ cpratio + rpm)

Residuals:
     Min       1Q   Median       3Q      Max 
-1323.67  -428.36   -78.54   227.83  2090.48 

Coefficients:
              Estimate Std. Error t value Pr(>|t|)    
(Intercept)  1.220e+04  3.809e+02  32.026  < 2e-16 ***
cpratio     -1.587e+02  2.103e+01  -7.548 2.02e-10 ***
rpm          1.446e-01  1.271e-02  11.383  < 2e-16 ***
---
Signif. codes:  0 ‘***’ 0.001 ‘**’ 0.01 ‘*’ 0.05 ‘.’ 0.1 ‘ ’ 1

Residual standard error: 631.9 on 64 degrees of freedom
Multiple R-squared:  0.8478,	Adjusted R-squared:  0.843 
F-statistic: 178.3 on 2 and 64 DF,  p-value: < 2.2e-16
\end{lstlisting}
We see that all terms are statistically significant and that the residuals have roughly the same distribution. 
\includegraphics[scale=0.5]{graphs/26b.png}
\\
Therefore the only suggested change to the model is to remove the interaction term with the final model being : \\
${{\hat y=\hat\beta_0+\hat\beta_1x_1}}$

\end{homeworkSection}
\end{homeworkProblem}


%---------------------------------------------
% Q 20
%---------------------------------------------
\begin{homeworkProblem}[Q 20]
%Question material goes here
Refer to the Journal of Engineering for Gas Turbines and
Power (January 2005) study of a high-pressure
inlet fogging method for a gas turbine engine,
Exercise 8.11 (p. 407). Use a residual graph to
check the assumption of normal errors for the
interaction model for heat rate (y). Is the normal-
ity assumption reasonably satisfied? If not, suggest
how to modify the model.

\begin{homeworkSection}{a}
	Here, we graph the residuals versus the expected residuals under the assumption of normality. If we assume normality, then we expect to see a near straight line from plotting th two against eachother. Failing to see this would be akin to a proof by contradiction. We assume the plot will be a straight line and see if the shows us otherwise. In this case, excluding the 3 extreme outliers, 11,36,64, this can be viewed as sufficiently normal.
	\includegraphics[scale=0.5]{graphs/26c.png}
	\\
	\includegraphics[scale=0.5]{graphs/20a.png}
	\\
	\includegraphics[scale=0.5]{graphs/20b.png}
	\\
	\includegraphics[scale=0.5]{graphs/20c.png}
	\\
	After excluding the 3 outliers in respect to residuals, the line appears resonably straight, and no values exceed the 3s limit, therefore we can assume that the constant variation of error is normal. The same model, with trimmed values is the suggested modified model. Doing analysis of the suggested model:
	\begin{lstlisting}
lm(formula = trimmedData$HEATRATE ~ trimmedData$CPRATIO + trimmedData$RPM, 
    data = trimmedData)

Residuals:
     Min       1Q   Median       3Q      Max 
-1150.79  -341.74     7.81   291.68  1138.87 

Coefficients:
                      Estimate Std. Error t value Pr(>|t|)    
(Intercept)          1.271e+04  3.789e+02  33.544  < 2e-16 ***
trimmedData$CPRATIO -1.939e+02  2.180e+01  -8.894 1.29e-12 ***
trimmedData$RPM      1.340e-01  1.058e-02  12.664  < 2e-16 ***
---
Signif. codes:  0 ‘***’ 0.001 ‘**’ 0.01 ‘*’ 0.05 ‘.’ 0.1 ‘ ’ 1

Residual standard error: 484.2 on 61 degrees of freedom
Multiple R-squared:  0.9003,	Adjusted R-squared:  0.897 
F-statistic: 275.3 on 2 and 61 DF,  p-value: < 2.2e-16
	\end{lstlisting}
	From the previous problem we can confirm that all terms of the model are statistically significant. Doing residual analysis:
	\\
	\includegraphics[scale=0.5]{graphs/20d.png}
	\\
	We can confirm that the distribution of the trimmed model is still normal and therefore we can suggest it as an alternative to the initial model. 
\end{homeworkSection}
\end{homeworkProblem}

%---------------------------------------------
% Q 26
%---------------------------------------------
\begin{homeworkProblem}[Q 26]
%Question material goes here
Prices of antique clocks. Refer to the grandfather
clock example, Example 4.1 (p. 183). The least
squares model used to predict auction price, y,from age of the clock, x 1 , and number of bidders,
x 2 , was determined to be
ŷ = −1, 339 + 12.74x 1 + 85.95x 2
\\(a) Use this equation to calculate the residuals of
each of the prices given in Table 4.1 (p. 171).
\\(b) Calculate the mean and the variance of the
residuals. The mean should equal 0, and the
variance should be close to the value of MSE
given in the SAS printout shown in Figure 4.3
(p. 172).
\\(c) Find the proportion of the residuals that fall
outside 2 estimated standard deviations (2s)
of 0 and outside 3s.
\\(d) Rerun the analysis and request influence diag-
nostics. Interpret the measures of influence
given on the printout.

\begin{homeworkSection}{a}
We can calculate the residuals plugging the appropriate values into the regression formula, then subtracting them from the actual values to get the difference. By fitting the same model in the question, R can do this very easily. The result is below. 
	\begin{lstlisting}
 -161.49037 -77.65049   -35.77246   176.28837 -117.29612   53.68403  141.98597   50.66572    116.91583  71.68961   17.19792   -73.17866   52.86881   -34.07440   -117.08463  210.10916  100.71196  -117.45702 213.49551   176.15586  -153.66866  22.62320  -185.37244  -33.27300  176.25143   76.81172   -189.54715 -138.15989  108.07673   16.11443 206.48496  -141.13600 
	\end{lstlisting}
\end{homeworkSection}

\begin{homeworkSection}{b}
	${{\mu_{residuals}}}$ = sum(residuals(model))/n =-2.442491e-15$\approx 0$
	\\
	Variance of model residuals = var(residuals(model)) = 16668.6
	\\
	Variance of actual Y's = var(GFCLOCKS\$PRICE)) = 154831.9
	\\
	Which is only a 10\% difference between the two variances.

\end{homeworkSection}

\begin{homeworkSection}{c}
	\includegraphics[scale=0.5]{graphs/26a.png}
	\\
	We see that graphically, no residuals fall outside 2s or 3s. To numerically find this proportion, we compare this to 2s and 3s.
	\\
We find that s = 133.48, 2s = 266.97, and 3s = 400.45. By checking the maximum of the residuals, we can see there are any residuals that fall outside 2s or 3s. We find that the maximum value for a residual from this model is 213.50 and we can then conclude that there is not a proportion of residuals that fall outside 2s or 3s. The absolute maximum of the residuals of the model is 213.50, which is not outside 2s or 3s, which means that the proportion of residuals outside of 2s or 3s is zero.
\end{homeworkSection}

\begin{homeworkSection}{d}
	Running the analysis with influence diagnostics:
	\begin{lstlisting}
	Influence measures of
	 lm(formula = GFCLOCKS$PRICE ~ GFCLOCKS$AGE + GFCLOCKS$NUMBIDS) :

     dfb.1_ dfb.GFCLOCKS.A dfb.GFCLOCKS.N   dffit cov.r   cook.d    hat inf
1   0.02869        0.08543       -0.26148 -0.3855 1.023 0.048488 0.0835    
2  -0.05862        0.10147       -0.06889 -0.1793 1.165 0.010954 0.0819    
3  -0.06946        0.04707        0.05651 -0.0832 1.203 0.002384 0.0836    
4   0.02387        0.03534       -0.03607  0.2519 0.948 0.020537 0.0330    
5  -0.10784       -0.01563        0.20211 -0.2722 1.107 0.024830 0.0814
6  -0.11675        0.12356        0.07164  0.1526 1.233 0.007988 0.1158    
7  -0.17188        0.13299        0.20699  0.3015 1.050 0.030061 0.0691    
8   0.03122       -0.03108        0.00327  0.0763 1.138 0.002001 0.0385    
9   0.08549       -0.05670       -0.04382  0.1725 1.061 0.009995 0.0363    
10  0.13709       -0.13037       -0.05189  0.1669 1.173 0.009508 0.0831    
11 -0.02517        0.00552        0.05138  0.0583 1.308 0.001172 0.1523    
12 -0.07287        0.09580       -0.02787 -0.1504 1.152 0.007724 0.0671    
13  0.05660       -0.03243       -0.04659  0.0887 1.146 0.002700 0.0469    
14 -0.03577        0.00108        0.05932 -0.0779 1.200 0.002088 0.0812    
15 -0.03668        0.12765       -0.17315 -0.3013 1.125 0.030415 0.0968    
16  0.19705       -0.20502       -0.00326  0.3673 0.879 0.042407 0.0467    
17 -0.29005        0.22077        0.30336  0.3738 1.247 0.047090 0.1707    
18  0.12457       -0.21815        0.03515 -0.2924 1.118 0.028655 0.0916    
19 -0.24865        0.24578        0.22009  0.4281 0.879 0.057341 0.0595    
20 -0.34324        0.40949        0.14699  0.4901 1.009 0.077356 0.1064    
21 -0.24163        0.09110        0.29752 -0.3731 1.042 0.045653 0.0863    
22 -0.00628        0.01501       -0.00183  0.0346 1.154 0.000414 0.0399    
23 -0.07864        0.29185       -0.37875 -0.6245 1.017 0.124219 0.1422    
24  0.02310       -0.04715        0.01280 -0.0712 1.190 0.001744 0.0724    
25  0.65137       -0.50561       -0.48185  0.6877 1.075 0.151355 0.1766    
26 -0.09600        0.13758        0.01477  0.1771 1.165 0.010692 0.0817    
27  0.02485        0.24471       -0.52572 -0.7250 1.030 0.166265 0.1703    
28  0.18473       -0.30244        0.03236 -0.3829 1.097 0.048507 0.1081    
29  0.31201       -0.25231       -0.20766  0.3356 1.181 0.037861 0.1309    
30  0.04201       -0.03291       -0.02916  0.0459 1.256 0.000728 0.1169    
31  0.13427       -0.50277        0.41400 -0.8280 0.985 0.212872 0.1796    
32  0.00229       -0.13043        0.14359 -0.3018 1.052 0.030135 0.0699
	\end{lstlisting}
	Abiding by $H > \frac{2(k+1)}{n}$ we find the threshold for an H score to be overly influential to the model to be 0.1875; which none of the H values on the right side of the graph exceed, which confirms the previous conclusion that there are no residual values that exceed the influence limit. 
\end{homeworkSection}
\end{homeworkProblem}

\end{document}
