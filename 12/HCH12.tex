%%%%%%%%%%%%%%%%%%%%%%%%%%%%%%%%%%%%%%%%%
% Structured General Purpose Assignment
% LaTeX Template
%
% This template has been downloaded from:
% http://www.latextemplates.com
%
% Original author:
% Ted Pavlic (http://www.tedpavlic.com)
%
% Note:
% The \lipsum[#] commands throughout this template generate dummy text
% to fill the template out. These commands should all be removed when 
% writing assignment content.
%
%%%%%%%%%%%%%%%%%%%%%%%%%%%%%%%%%%%%%%%%%

%-------------------z---------------------------------------------------------------------
%	PACKAGES AND OTHER DOCUMENT CONFIGURATIONS
%----------------------------------------------------------------------------------------

\documentclass{article}

\usepackage{fancyhdr} % Required for custom headers
\usepackage{lastpage} % Required to determine the last page for the footer
\usepackage{extramarks} % Required for headers and footers
\usepackage{graphicx} % Required to insert images

\usepackage{lipsum} % Used for inserting dummy 'Lorem ipsum' text into the template
\usepackage{amsmath}
\usepackage{verbatim}

\usepackage{listings}
\usepackage{xcolor}

\lstset{
  basicstyle=\ttfamily,
  escapeinside=||
}
% Margins
\topmargin=-0.45in{}
\evensidemargin=0in
\oddsidemargin=0in
\textwidth=6.5in
\textheight=9.0in
\headsep=0.25in 

\linespread{1.1} % Line spacing

% Set up the header and footer
\pagestyle{fancy}
\lhead{\hmwkAuthorName} % Top left header
\chead{\hmwkClass\ (\hmwkClassInstructor\ \hmwkClassTime): \hmwkTitle} % Top center header
\rhead{\firstxmark} % Top right header
\lfoot{\lastxmark} % Bottom left footer
\cfoot{} % Bottom center footer
\rfoot{Page\ \thepage\ of\ \pageref{LastPage}} % Bottom right footer
\renewcommand\headrulewidth{0.4pt} % Size of the header rule
\renewcommand\footrulewidth{0.4pt} % Size of the footer rule

\setlength\parindent{0pt} % Removes all indentation from paragraphs

%----------------------------------------------------------------------------------------
%	DOCUMENT STRUCTURE COMMANDS
%	Skip this unless you know what you're doing
%----------------------------------------------------------------------------------------

% Header and footer for when a page split occurs within a problem environment
\newcommand{\enterProblemHeader}[1]{
\nobreak\extramarks{#1}{#1 continued on next page\ldots}\nobreak
\nobreak\extramarks{#1 (continued)}{#1 continued on next page\ldots}\nobreak
}

% Header and footer for when a page split occurs between problem environments
\newcommand{\exitProblemHeader}[1]{
\nobreak\extramarks{#1 (continued)}{#1 continued on next page\ldots}\nobreak
\nobreak\extramarks{#1}{}\nobreak
}

\setcounter{secnumdepth}{0} % Removes default section numbers
\newcounter{homeworkProblemCounter} % Creates a counter to keep track of the number of problems


% PROBLEM 8
\newcommand{\homeworkProblemName}{}
\newenvironment{homeworkProblem}[1][Problem \arabic{homeworkProblemCounter}]{ % Makes a new environment called homeworkProblem which takes 1 argument (custom name) but the default is "Problem #"
\stepcounter{homeworkProblemCounter} % Increase counter for number of problems
\renewcommand{\homeworkProblemName}{#1} % Assign \homeworkProblemName the name of the problem
\section{\homeworkProblemName} % Make a section in the document with the custom problem count
\enterProblemHeader{\homeworkProblemName} % Header and footer within the environment
}{
\exitProblemHeader{\homeworkProblemName} % Header and footer after the environment
}

\newcommand{\problemAnswer}[1]{ % Defines the problem answer command with the content as the only argument
\noindent\framebox[\columnwidth][c]{\begin{minipage}{0.98\columnwidth}#1\end{minipage}} % Makes the box around the problem answer and puts the content inside
}

\newcommand{\homeworkSectionName}{}
\newenvironment{homeworkSection}[1]{ % New environment for sections within homework problems, takes 1 argument - the name of the section
\renewcommand{\homeworkSectionName}{#1} % Assign \homeworkSectionName to the name of the section from the environment argument
\subsection{\homeworkSectionName} % Make a subsection with the custom name of the subsection
\enterProblemHeader{\homeworkProblemName\ [\homeworkSectionName]} % Header and footer within the environment
}{
\enterProblemHeader{\homeworkProblemName} % Header and footer after the environment
}
   
%----------------------------------------------------------------------------------------
%	NAME AND CLASS SECTION
%----------------------------------------------------------------------------------------

\newcommand{\hmwkTitle}{Homework Chapter 1: 
\\Analysis of Variance} % Assignment title
\newcommand{\hmwkDueDate}{Thursday,\ December\ 14,\ 2017} % Due date
\newcommand{\hmwkClass}{MATH\ 424} % Course/class
\newcommand{\hmwkClassTime}{11:10am} % Class/lecture time
\newcommand{\hmwkClassInstructor}{Kafai} % Teacher/lecturer
\newcommand{\hmwkAuthorName}{Jonathan Dombrowski} % Your name

%----------------------------------------------------------------------------------------
%	TITLE PAGE
%----------------------------------------------------------------------------------------

\title{
\vspace{2in}
\textmd{\textbf{\hmwkClass:\ \hmwkTitle}}\\
\normalsize\vspace{0.1in}\small{Due\ on\ \hmwkDueDate}\\
\vspace{0.1in}\large{\textit{\hmwkClassInstructor\ \hmwkClassTime}}
\vspace{3in}
}

\author{\textbf{\hmwkAuthorName}}
\date{} % Insert date here if you want it to appear below your name

%----------------------------------------------------------------------------------------

\begin{document}

\maketitle

%----------------------------------------------------------------------------------------
%	TABLE OF CONTENTS
%----------------------------------------------------------------------------------------

%\setcounter{tocdepth}{1} % Uncomment this line if you don't want subsections listed in the ToC

\newpage
\tableofcontents
\newpage

%----------------------------------------------------------------------------------------
%---------------------------------------------
% Q 
%---------------------------------------------
\begin{homeworkProblem}[Q ]
%Question material goes here
IS there sufficient evidence to indicate the differences among the mean Al/Be ratios for the five boreholes? Test using $\alpha$ = 0.10.

\begin{homeworkSection}{a}
	Model to test : ${{E(y) =\beta_0 = \beta_1x_1}}$\\
	Creating dummy variables for the Borehole type, the model changes to :
	\[
      E(y)= \beta_0 +\beta_1x_1 + \beta_2x_2 + \beta_3x_3 + \beta_4x_4
	\]
		\[
		\beta_{0}
		\bigg\{
		  \begin{tabular}{ll}
		  0 :  \text{if not}\\
		  1 :  \text{SD}
		  \end{tabular}
		\]

		\[
		x_{1}
		\bigg\{
		  \begin{tabular}{ll}
		  0 :  \text{if not}\\
		  1 :  \text{SWRA}
		  \end{tabular}
		\]

		\[
		x_{2}
		\bigg\{
		  \begin{tabular}{ll}
		  0 :  \text{if not}\\
		  1 :  \text{URMB-1}
		  \end{tabular}
		\]
		\[
		x_{3}
		\bigg\{
		  \begin{tabular}{ll}
		  0 :  \text{if not}\\
		  1 :  \text{URMB-2}
		  \end{tabular}
		\]
		\[
		x_{4}
		\bigg\{
		  \begin{tabular}{ll}
		  0 :  \text{if not}\\
		  1 :  \text{URMB-3}
		  \end{tabular}
		\]

Testing for overall model validity
\[
	H_o = \beta_1 = \beta_2 =\beta_3 = \beta_4 = 0
\]
\[
	H_a = \beta_i \neq \beta_j , \forall i,j : i\neq j
\]	
\begin{lstlisting}	
Call:
glm(formula = RATIO ~ BOREHOLE, data = TILLRATIO)

Deviance Residuals: 
    Min       1Q   Median       3Q      Max  
-0.6967  -0.2821  -0.1267   0.2939   1.0433  

Coefficients:
               Estimate Std. Error t value Pr(>|t|)    
(Intercept)      2.7800     0.2590  10.734 5.52e-10 ***
BOREHOLESWRA    -0.1367     0.3663  -0.373 0.712778    
BOREHOLEUMRB-1   0.7171     0.3095   2.317 0.030701 *  
BOREHOLEUMRB-2   1.2367     0.3172   3.899 0.000827 ***
BOREHOLEUMRB-3   0.9814     0.3095   3.171 0.004607 ** 
---
Signif. codes:  
0 ‘***’ 0.001 ‘**’ 0.01 ‘*’ 0.05 ‘.’ 0.1 ‘ ’ 1

(Dispersion parameter for gaussian family taken to be 0.2012109)

    Null deviance: 10.0612  on 25  degrees of freedom
Residual deviance:  4.2254  on 21  degrees of freedom
AIC: 38.543
	\end{lstlisting}	
	Calling analysis of variance on the model gives us the p-value we are looking for:
\begin{lstlisting}
            Df Sum Sq Mean Sq F value   Pr(>F)    
BOREHOLE     4  5.836  1.4589   7.251 0.000784 ***
Residuals   21  4.225  0.2012                     
---
Signif. codes:  
0 ‘***’ 0.001 ‘**’ 0.01 ‘*’ 0.05 ‘.’ 0.1 ‘ ’ 1
	\end{lstlisting}
From this we can complete the test and since $\alpha = 0.10$ is greater than the p-value of 0.000784, therefore we reject the null hypothesis and state that there is sufficient evidence to sugget that there are differences between the mean Al/Be ratios for the five boreholes. The visual below supports these claims.  

\includegraphics[scale=0.25]{graphs/16boxplot.png}
\end{homeworkSection}


\end{homeworkProblem}













%---------------------------------------------
% Q 20
%---------------------------------------------
\begin{homeworkProblem}[Q 20]
%Question material goes here
(a) Explain why this data should be analyzed
using a randomized block design. As part
of your answer, identify the blocks and the
treatments.
\\
(b) A partial ANOVA table for the experiment
is shown below. Explain why there is enough
information in the table to make conclusions.
\\
(c) State the null hypothesis of interest to the
researcher.
\\
(d) Make the appropriate conclusion.
\begin{homeworkSection}{a}
	\problemAnswer{
		The Randomized Block Design helps remove biases in analyzing data with factors that can influence the main factors, but are not the main focus of the experiment. In this case, we expect that the scores at different times will be more alike than the the total comparison. The goal of the experiment was to compare the mean competence levels of the three periods. In order to abstract the bias of different persons' starting scores, we can block the experiment in such a way that the 222 participants are the different blocks and the treatments are the three time periods: 1wk before, 2d after and 2m after. 
	}
\end{homeworkSection}

\begin{homeworkSection}{b}
	\problemAnswer{
		Being able to make a conclusion is dependent on accepting or rejecting a hypothesis based on a F-test/p-value. In order to find that information, we have to calculate various terms from the data. If we are already given the p-values, then the other information is unnecessary in making a reasonable conclusion. Therefore we are given enough information to make conclusions. 
	}
\end{homeworkSection}

\begin{homeworkSection}{c}
	\problemAnswer{
	The researcher is interested in comparing the mean competence levels of the three periods. This relies on testing whether or not the means are all the same. The null hypothesis in the interest of the researchers is as follows:
		\[
			H_o = \mu_1 = \mu_2 = \mu _3 = 0
		\]
		with the accompanying alternative hypothesis being:
		\[
			H_a = \mu_i \neq \mu_; \forall i\neq j
		\]	
	}
\end{homeworkSection}

\begin{homeworkSection}{d}
	\problemAnswer{ 
		Based on the partial ANOVA table given to us in the text, we can reject the null hypothesis in the previous question and state that there is at least one pair of $\mu$'s that are different therefore the time period has an effect on determining the competence level. If we are trying to build a model to predict the competency of trainees, then taking the 222 different trainees into account is going to be a significant factor in reducing the variance in the model. 
		}

\end{homeworkSection}
	
\end{homeworkProblem}





















%---------------------------------------------
% Q 26
%---------------------------------------------
\begin{homeworkProblem}[Q 26]
Massage therapy for boxers (cont’d). Refer to
Exercise 12.25. The models of parts a, b, and c
were fit to data in the table using MINITAB. The
MINITAB printouts are displayed here.
\\
(a) Construct an ANOVA summary table.
\\
(b) Is there evidence of differences in the punch-
ing power means of the four interventions?
Use α = .05.
\\
(c) Is there evidence of a difference among the
punching power means of the boxers? That
is, is there evidence that blocking by box-
ers was effective in removing an unwanted
source of variability? Use α = .05.

\begin{homeworkSection}{a}
From the General format of an ANOVA table for Randomize Block Design we can see the formulas for calculating the Anova Table. 
\begin{lstlisting}
Source           df        SS       MS                 F
Treatments       p - 1     SST   MST = SST/(p-1)       F = MST/MSE
Blocks           b - 1     SSB   MSB = SSB/(b-1)       F = MST/MSE
Error    n - p - b + 1     SSE   MSE=SSE/(n-p-b+1)
Total            n - 1     SS(Total) 
\end{lstlisting}


Followed by the filled table with the values from the text


		\begin{lstlisting}
---------------------------------------------------------------------
              d.f.      SS       MS     F(ratio)   p-value
Treatments      3     15,754    5,251     4.158    0.133  
Block           7    117,044   16,721    13.230    0.00092
Error          21     26,525    1,263     
Total          31    159,323   13,280    
		\end{lstlisting} 

\end{homeworkSection}

\begin{homeworkSection}{b}
	\problemAnswer{
		We are checking the following test for the reduced model of just the Treatments consisting of the different times for the massages. 
		\[
			H_o =  I_0 = I_1 = I_2 = I_3=0
		\]
		\[
			H_a = I_i \neq I_j ; \forall i \neq j
		\]
		Using the F-test we derived in the Anova table:
		\[
			F= \frac{MST}{MSE} = 4.158
		\]
		The accompanying p-value for this F-score is : 0.133.
		\\
		Since we are using $\alpha=0.05$, then we can conclude that p is greater than $\alpha$ and therefore we fail to reject the null hypothesis. There is not substantial evidence to show differences in the punching power means of the four interventions. 
	}
\end{homeworkSection}
\begin{homeworkSection}{c}
	\problemAnswer{
		We are checking the following test for the reduced model of just the Blocks consisting of the different boxers.
		\[
			H_o =\beta_0 = \beta_1 = \beta_2 = \ldots = \beta_7=0
		\]
		\[
			H_a = \beta_i \neq \beta_j ; \forall i \neq j
		\]
		Using the F-test we derived in the Anova table:
		\[
			F= \frac{MST}{MSE} = 13.230
		\]
		The accompanying p-value for this F-score is : 0.00092.
		\\
		Since we are using $\alpha=0.05$, then we can conclude that p is less than $\alpha$. Therefore we can conclude that the blocking by boxers is effective in removing unwanted variability. We can conclude that the mean punching power of a boxer is dependant on which boxer is the individual in question. This conclusion of dependency is further shown in the below tables
	}
	\includegraphics[scale = 0.45]{graphs/26Boxer.png}
	\\
	\includegraphics[scale = 0.45]{graphs/26Round.png}
\end{homeworkSection}

\end{homeworkProblem}


























%---------------------------------------------
% Q 36
%---------------------------------------------
\begin{homeworkProblem}[Q 36]
%Question material goes here
Commercial eggs produced from different hous-
ing systems. In the production of commercial
eggs in Europe, four different types of housing
systems for the chickens are used: cage, barn, free
range, and organic. The characteristics of eggs
produced from the four housing systems were
investigated in Food Chemistry (Vol. 106, 2008).
Twenty-four commercial grade A eggs were ran-
domly selected—six from each of the four types
of housing systems. Of the six eggs selected from
each housing system, three were Medium weight
class eggs and three were Large weight class eggs.
The data on whipping capacity (percent overrun)
for the 24 sampled eggs are shown in the next
table. The researchers want to investigate the
effect of both housing system and weight class on
the mean whipping capacity of the eggs. In par-
ticular, they want to know whether the difference
between the mean whipping capacity of medium
and large eggs depends on the housing system.
\\
(a) Identify the factors and treatments for this
experiment.
\\
(b) Use statistical software to conduct an
ANOVA on the data. Report the results
in an ANOVA table.
\\
(c) Is there evidence of interaction between
housing system and weight class? Test using
α = .05. What does this imply, practically?

\begin{homeworkSection}{a}
	\problemAnswer{
	%Question answer goes here
	There are two factors in this experiment: Housing and Weight Class. The treatments for this are the \{Medium,Large\} * \{Cage, Barn, Free-Range, Organic\} for a total of 8 treatments.
	}
\end{homeworkSection}

\begin{homeworkSection}{b}
Fitting the model onto the data we get the following readout:

\begin{lstlisting}
Call:
lm(formula = OVERRUN ~ HOUSING + WTCLASS + HOUSING:WTCLASS, data = EGGS2)

Residuals:
   Min     1Q Median     3Q    Max 
-19.67  -4.25  -1.00   6.50  19.33 

Coefficients:
                                Estimate Std. Error t value Pr(>|t|)    
(Intercept)                      511.667      6.365  80.387  < 2e-16 ***
HOUSINGCAGE                      -29.000      9.002  -3.222  0.00533 ** 
HOUSINGFREE                       12.333      9.002   1.370  0.18956    
HOUSINGORGANIC                    23.333      9.002   2.592  0.01965 *  
WTCLASSM                           3.333      9.002   0.370  0.71601    
HOUSINGCAGE    :WTCLASSM          -4.333     12.730  -0.340  0.73798    
HOUSINGFREE    :WTCLASSM         -16.333     12.730  -1.283  0.21775    
HOUSINGORGANIC :WTCLASSM         -15.000     12.730  -1.178  0.25590    
---
Signif. codes:  0 ‘***’ 0.001 ‘**’ 0.01 ‘*’ 0.05 ‘.’ 0.1 ‘ ’ 1

Residual standard error: 11.02 on 16 degrees of freedom
Multiple R-squared:  0.7989,	Adjusted R-squared:  0.7109 
F-statistic:  9.08 on 7 and 16 DF,  p-value: 0.0001454
\end{lstlisting}

With the accompanying ANOVA table:
	\begin{lstlisting}
                Df Sum Sq Mean Sq F value  Pr(>F)    
HOUSING          3   7249  2416.5  19.882 1.2e-05 ***
WTCLASS          1    187   187.0   1.539   0.233    
HOUSING:WTCLASS  3    289    96.3   0.792   0.516    
Residuals       16   1945   121.5                    
---
Signif. codes:  0 ‘***’ 0.001 ‘**’ 0.01 ‘*’ 0.05 ‘.’ 0.1 ‘ ’ 1
	\end{lstlisting}
	The WTCLASS seems to have no effect in determining the percent Overrun. This is further confirmed by the following boxplot. 
	\\
	\includegraphics[scale= 0.25]{graphs/36boxWT.png}f
\end{homeworkSection}

\begin{homeworkSection}{c}
	\problemAnswer{
		Testing whether or not the HOUSING and WTCLASS factors have any interaction:
		\[
		H_o : \beta_3 = 0 
		\]
		\[
		H_a : \beta_3 \neq 0 
		\]
		Using the output from the ANOVA table above, we can see that the p-value for the interaction variables is 0.516. From this we can see that p is far greater than $\alpha$ and can fail to reject $H_{0}$. From this we can conclude that there is no interaction between the HOUSING and WTCLASS in predicting OVERRUN. 
	}
	\includegraphics[scale=0.25]{graphs/36boxHouse.png}
\end{homeworkSection}

\end{homeworkProblem}










%---------------------------------------------
% Q 56
%---------------------------------------------
\begin{homeworkProblem}[Q 56]
%Question material goes here
Commercial eggs produced from different hous-
ing systems. Refer to the Food Chemistry (Vol.
106, 2008) study of four different types of egg
housing systems, Exercise 12.36 (p. 659). Recall
that you discovered that the mean whipping
capacity (percent overflow) differed for cage,
barn, free range, and organic egg housing systems.
A multiple comparisons of means was conducted
using Tukey’s method with an experimentwise error rate of .05. The results are displayed in the
SPSS printout above.
\\
(a) Locate the confidence interval for (l CAGE −
μ BARN ) on the printout and interpret the
result.
\\
(b) Locate the confidence interval for (μ CAGE −
μ FREE ) on the printout and interpret the
result.
\\
(c) Locate the confidence interval for (μ CAGE −
μ ORGANIC ) on the printout and interpret the
result.
\\
(d) Locate the confidence interval for (μ BARN −
μ FREE ) on the printout and interpret the
result.
\\
(e) Locate the confidence interval for (μ BARN −
μ ORGANIC ) on the printout and interpret the
result.
\\
(f) Locate the confidence interval for (μ FREE −
μ ORGANIC ) on the printout and interpret the
result.
\\
(g) Based on the results, parts a–f, provide a
ranking of the housing system means. Include
the experimentwise error rate as a statement
of reliability.

\begin{homeworkSection}{a}
	\problemAnswer{
		The CI for $\mu_{cage}-\mu_{barn}$ is (-49.38, -12.95). The true difference between the cage raised and the barn raised percent overflow will lie within the interval of (-49.38, -12.95) for 95\% of samples from the population. This difference is statistically significant with a significance level lower than the threshold: 0.001 vs. 0.05; therefore the pair of treatments has a decisive difference. Barn raised eggs have a significantly higher percentage overrun in regards to cage raised.

	}
\end{homeworkSection}

\begin{homeworkSection}{b}
	\problemAnswer{
		The CI for $\mu_{cage}-\mu_{free}$ is (-53.54,-17.12). The difference between cage raised and the free range percent overflow will lie within the interval of (-53.54,-17.12) for 95\% of the samples from the population. This difference is statistically significant with a significance level lower than the threshold: .000 vs. 0.05; therefore cage-raised chickens have a decisively lower percent overrun than free-range. 
	
	}
\end{homeworkSection}

\begin{homeworkSection}{c}
	\problemAnswer{
		The CI for $\mu_{cage}-\mu_{organic}$ is (-65.21,-28.79). The difference between the cage and organic percent overflow will lie withing the interval (-65.21,-28.79) 95\% of the time. This difference is statistically significant with a significance level lower than the threshold: .000 vs. 0.05. Therefore the cage raised chickens have a decisively lower percent overrun than the organically raised. 
	}
\end{homeworkSection}

\begin{homeworkSection}{d}
	\problemAnswer{
		The CI for $\mu_{barn}-\mu_{free}$ is (-22.38,14.04). The difference between the barn and free percentage overflow will lie in the interval (-22.38,-17.12) 95\%  of the samples from the population. The mean difference is NOT statisticaly significant with a significance level of 0.912 vs. 0.05. Therefore we cannot reject the hypothesis that the two means are different. 
	}
\end{homeworkSection}

\begin{homeworkSection}{e}
	\problemAnswer{
		The CI for $\mu_{barn}-\mu_{organic}$ is (-34.04,2.38). The difference between the barn and the organic percentage overflow lie in the interval (-34.04,2.38) 95\%  of the samples from the population. The mean difference is NOT statisticaly significant with a significance level of 0.100 vs. 0.05. Therefore we cannot reject the hypothesis that the two means are different. 
	}
\end{homeworkSection}

\begin{homeworkSection}{f}
	\problemAnswer{
		The CI for $\mu_{free}-\mu_{organic}$ is (-29.88,6.54). The difference between the free and the organic percentage overflow wil lie between the interval (-29.88,6.54) 95\% of the samples from the population. The mean difference is NOT statisticaly significant with a significance level of 0.295 vs. 0.05. Therefore we cannot reject the hypothesis that the two means are different. 
	}
\end{homeworkSection}

\begin{homeworkSection}{g}
Ordering the Tukey readout from R gives us the following table, which orders the difference of th means from highest to lowest, and also from lowest p-value to highest. The p-values in the table are significant at $\alpha=$ 0.05.
	\begin{lstlisting}
  Tukey multiple comparisons of means
    95 percent family-wise confidence level

Fit: aov(formula = modelA)

$HOUSING
                        diff        lwr       upr     p adj
ORGANIC -CAGE      47.000000  29.222532  64.77747 0.0000021
FREE    -CAGE      35.333333  17.555865  53.11080 0.0001048                        
CAGE    -BARN     -31.166667 -48.944135 -13.38920 0.0004576
-----------------------------------------------------------
ORGANIC -BARN      15.833333  -1.944135  33.61080 0.0917490
ORGANIC -FREE      11.666667  -6.110801  29.44413 0.2861315
FREE    -BARN       4.166667 -13.610801  21.94413 0.9121909
	\end{lstlisting}
\end{homeworkSection}



\end{homeworkProblem}













%---------------------------------------------
% Q 65
%---------------------------------------------
\begin{homeworkProblem}[Q 65]
%Question material goes here

\begin{homeworkSection}{a}
The R printout is at the bottom of this, but for proof of concept, verifying the method used in R seemed appropriate. The final R printout is below the proof of concept. 

Using Tukeys' comparison method: 
\[
	w_{ij}= q_{\alpha}(p,v)\sqrt{\frac{MSE}{2}}*\sqrt{\frac{1}{n_i}+\frac{1}{n_j}}
\]

$q_{\alpha}(p,v)=q_{\alpha}(5,21)$ = 3.578

\[
	w= 3.578*\sqrt{\frac{0.2012}{2}}*\sqrt{\frac{1}{n_i}+\frac{1}{n_j}}
\]

\[
	= 1.117*\sqrt{\frac{1}{n_i}+\frac{1}{n_j}}
\]
In order to calculate the $w_{{ij}}$ we have to have the specific treatments in mind. Testing this on two different pairs:

\[
	w_{URMB-2/SD}= 1.117*\sqrt{\frac{1}{6}+\frac{1}{3}}=0.7898
\]

\[
	w_{SWRA/SD}= 1.117*\sqrt{\frac{1}{7}+\frac{1}{3}}=0.7710
\]

The distance in the means is greater than the w for URMB-2/SD (0.7898 vs. 1.23) and less than for the SWRA/SD pair (0.771 vs. -0.0133). From this we can conclude that the difference in means for URMB-2/SD is statistically significant and the means for SWRA/SD are not statistically significant. This reaffirms the p-values in the following R-output table ran from the built in Tukey multipe analysis. 

\begin{lstlisting}  Tukey multiple comparisons of means
    90 percent family-wise confidence level

Fit: aov(formula = model)

$BOREHOLE
                    diff         lwr       upr     p adj
SWRA  -SD     -0.1366667 -1.10100241 0.8276691 0.9955440
UMRB-1-SD      0.7171429 -0.09786960 1.5321553 0.1788060
UMRB-2-SD      1.2366667  0.40152741 2.0718059 |\colorbox{magenta!30}{0.0066199}|
UMRB-3-SD      0.9814286  0.16641611 1.7964410 |\colorbox{magenta!30}{0.0333957}|
UMRB-1-SWRA    0.8538095  0.03879707 1.6688220 |\colorbox{magenta!30}{0.0782702}|
UMRB-2-SWRA    1.3733333  0.53819408 2.2084726 |\colorbox{magenta!30}{0.0024546}|
UMRB-3-SWRA    1.1180952  0.30308278 1.9331077 |\colorbox{magenta!30}{0.0126734}|
UMRB-2-UMRB-1  0.5195238 -0.13756024 1.1766079 0.2643848
UMRB-3-UMRB-1  0.2642857 -0.36702022 0.8955916 0.8034378
UMRB-3-UMRB-2 -0.2552381 -0.91232215 0.4018460 0.8420582
\end{lstlisting}

Further examination of the p-values listed in the table show that the difference of the means of the highlighted pairs are statistically significant through the use of the Tukey Multiple Comparisons method at an experiment wise error rate of 0.1. These pairs will have decisive impact on the prediction of glacial drift age.

This includes:
\begin{itemize}
\item UMRB-2-SD
\item UMRB-3-SD
\item UMRB-1-SWRA 
\item UMRB-2-SWRA 
\item UMRB-3-SWRA 
\end{itemize}
To graphically support this conclusion:
\\
\includegraphics[scale = 0.25]{graphs/65box.png}

\end{homeworkSection}

\end{homeworkProblem}
\end{document}
	